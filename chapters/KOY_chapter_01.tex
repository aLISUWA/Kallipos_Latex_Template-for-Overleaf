\chapter{Γενικά χαρακτηριστικά και δομή περιεχομένου}\label{chap:character}
\chapterauthor{Κεντρική Ομάδα Υποστήριξης}[Κεντρική Ομάδα Υποστήριξης \\ Κάλλιπος, Ανοικτές Ακαδημαϊκές Εκδόσεις]
\section{Κατηγορίες ηλεκτρονικών βιβλίων }
Στο Έργο ΚΑΛΛΙΠΟΣ+ χρηματοδοτούνται βιβλία/συγγράμματα \footnote{Εφεξής ο όρος θα χρησιμοποιείται εναλλάξ.} τα οποία -από πλευράς
περιεχομένου- ανήκουν σε μία από τις επόμενες κατηγορίες:
\begin{enumerate}
\item \emph{Εγχειρίδια} για Προπτυχιακά \& Μεταπτυχιακά μαθήματα (Νέα συγγράμματα)
\item \emph{Μεταφράσεις} ανοικτών ξενόγλωσσων συγγραμμάτων (open textbooks)
\item \emph{Μονογραφίες} (Νέα συγγράμματα)
\item \emph{Βιβλιογραφικοί Οδηγοί} (Νέα συγγράμματα)
\end{enumerate}

Τα τεχνικά χαρακτηριστικά  που περιγράφονται στις επόμενες παραγράφους αφορούν
κυρίως την ανάπτυξη συγγραμμάτων στις κατηγορίες 1 (Εγχειρίδια) και 3 (Μονογραφίες).
Στις Mεταφράσεις ξενόγλωσσων συγγραμμάτων (textbooks) είναι δυνατόν να ακολουθείται
η δομή και ο μορφότυπος του πρωτότυπου συγγράμματος, ενώ για τους Βιβλιογραφικούς
Οδηγούς, αν και απλούστερο -ως προς τη δομή- είδος συγγραμμάτων, μπορεί να υιοθετείται
o μορφότυπος των κατηγοριών 1 και 3.

\section{Δομή περιεχομένου}
Όπως  καταγράφεται  και  στο  Τεχνικό  Δελτίο  του  Έργου,  στο  πλαίσιο  του  Έργου
ΚΑΛΛΙΠΟΣ+  καλούνται  οι  υποψήφιοι  Συγγραφείς  να  παραδώσουν  συγγράμματα  σε
ηλεκτρονική μορφή με πρωταρχικό  στόχο την αξιοποίησή τους στη διδασκαλία  στα
ελληνικά  Ανώτατα  Εκπαιδευτικά  Ιδρύματα και  στα  Ανώτατα  Στρατιωτικά
Εκπαιδευτικά  Ιδρύματα  της  χώρας  (κυρίως  κατηγορίες  1,  2 και  3  προηγούμενης
παραγράφου). Επομένως, η έκταση του περιεχομένου τους και η διάρθρωσή τους θα πρέπει
να ικανοποιεί τις ανάγκες για τη διεξαγωγή ενός -τουλάχιστον εξαμηνιαίου-  μαθήματος.
Με αφετηρία τα ως άνω, στον Πίνακα \ref{table:megethi} που ακολουθεί παρατίθενται (ως κατάλληλα) τα
\emph{ενδεικτικά} μεσοσταθμικά μεγέθη που θα πρέπει να τηρεί ένα προτεινόμενο ηλεκτρονικό
βιβλίο.

\begin{table} [h] \centering
\caption{Ενδεικτικά μεσοσταθμικά μεγέθη περιεχομένου.}
\vspace{2mm}
\begin{tabular} {l c c}
\hline
	\textbf{Κατηγορία}	&\textbf{Όρια}	&\textbf{Μέσος όρος}\\
\hline
		&	    &	\\
Αριθμός κεφαλαίων	&5 -15 	&10\\
Αριθμός σελίδων	&250 - 350	    &300	\\
Αριθμός σελίδων ανά κεφάλαιο	&25- 35	    &30	\\
	&	    &	\\
\hline
\end{tabular}
\label{table:megethi}
\end{table}

Η γενική δομή ενός εγχειριδίου (προτείνεται να) διαρθρώνεται ως εξής:
\begin{itemize}
\item Ενότητα  1  - Πρώτες  σελίδες  (front pages): Σελίδα  τίτλου,  σελίδα  ISBN  και
Συντελεστών, σελίδα αφιέρωσης (προαιρετική).
\item Ενότητα 2 – Περιεχόμενα (υποχρεωτικό): Πίνακας περιεχομένων.
\item  Ενότητα 3 – Πρόλογος (προαιρετικό): Σημείωμα από τη Συγγραφική Ομάδα ή τρίτα
πρόσωπα.
\item Ενότητα 4 –  Εισαγωγή (προαιρετικό):  Εισαγωγικό σημείωμα από τη Συγγραφική
Ομάδα.
\item  Ενότητα 5 – Κεφάλαια [συμπεριλαμβανομένων, ανά κεφάλαιο, των βιβλιογραφικών
αναφορών (υπο\-χρε\-ω\-τι\-κά) και (προαιρετικά) της θεματικής/επιστημονικής ορολογίας].
\item  Ενότητα 6 – Λίστα σημαντικών μαθησιακών αντικειμένων (προαιρετικό).
\end{itemize}

Όσον αφορά την Ενότητα 5, και ειδικά για την κατηγορία των \emph{Εγχειριδίων για Προπτυχιακά
μαθήματα}, κάθε Κεφάλαιο θα πρέπει να αντιστοιχεί σε μία έως δύο διδακτικές εβδομάδες
ή, αν πρόκειται για Εργαστηριακούς Οδηγούς, να αντιστοιχεί στην εκτέλεση τουλάχιστον
μίας Εργαστηριακής Άσκησης. Επίσης, ένα Κεφάλαιο (και κατά προέκταση το βιβλίο) θα
πρέπει  να  οργανώνεται  στη  μορφή  ενός  ή  περισσοτέρων,  αυτοτελών  στο  μέτρο  του
δυνατού, αντικειμένων  περιεχομένου  (μαθησιακών/εκπαιδευτικών  αντικειμένων).  Οι
παρακάτω κατηγορίες είναι ενδεικτικές, κυρίως σε ό, τι αφορά τα προαιρετικά στοιχεία των
μαθησιακών αντικειμένων:

\begin{table} [h] \centering
\caption{Κατηγορίες περιεχομένου ανά Κεφάλαιο.}
\vspace{2mm}
\begin{tabular} {p{0.55\linewidth} p{0.15\linewidth} p{0.15\linewidth}}
\hline
	\textbf{Πεδίο}	&\textbf{Υποχρεωτικό}	&\textbf{Προαιρετικό}\tabularnewline
\hline
		&	    &	\tabularnewline
Τίτλος		& \centering \centering $\surd$	    &\tabularnewline
Σύνοψη – Περίληψη	&\centering $\surd$	    	&	\tabularnewline
Προαπαιτούμενη γνώση (αναφορές σε άλλα κεφάλαια / βιβλία ή σε λήμματα από καθιερωμένα λεξιλόγια)& \centering $\surd$ &	\tabularnewline
Κυρίως κείμενο	&\centering $\surd$	    &	\tabularnewline
Βιβλιογραφία (Αναφορές/References, και παραπομπές εντός του κειμένου / in-text citations)	&\centering $\surd$	    &	\tabularnewline
Γλωσσάριο/-α επιστημονικών όρων (στην αρχή ή/και στο τέλος του κεφαλαίου) [πχ. πίνακας συντομεύσεων - ακρωνυμίων (στην ελληνική και αγγλική γλώσσα), «ευρετήριο» θεματικών όρων πεδίου]	&	    &\centering $\surd$	\tabularnewline
Προσδοκώμενα μαθησιακά αποτελέσματα/στόχοι	&	    &\centering $\surd$	\tabularnewline
Κριτήρια αξιολόγησης (πχ. φύλλα ερωτήσεων/ασκήσεων αυτοαξιολόγησης/προβλημάτων) με ενδεικτικές απαντήσεις- λύσεις	&	    &\centering $\surd$	\tabularnewline
	&	    &	\tabularnewline
\textbf{Αντικείμενα εντός του κειμένου} &	&		\tabularnewline
\hline
	&	    &	\tabularnewline
Πίνακες	&	    &\centering $\surd$	\tabularnewline
Σχήματα – χάρτες (απλά)	&	    &\centering $\surd$	\tabularnewline
Μαθηματικά αντικείμενα-σύμβολα (σύνολα μαθηματικών ή λογικών σχέσεων)	&	    &\centering $\surd$	\tabularnewline
Αλγοριθμικά αντικείμενα - (ψευδο)κώδικας	&	    &\centering $\surd$	\tabularnewline
	&	    &	\tabularnewline
\hline
\end{tabular}
\label{table:content}
\end{table}

\section{Μεταδεδομένα}
Τα μεταδεδομένα αποτελούν πληροφορία που περιγράφει τα δεδομένα. Ιδιαίτερη βαρύτητα
δίνεται στη συμπλήρωση των μεταδεδομένων (στην ελληνική και στην αγγλική γλώσσα),
τόσο σε επίπεδο βιβλίου όσο και σε επίπεδο Κεφαλαίου, αλλά και στα λοιπά  μαθησιακά
αντικείμενα  που  συγκροτούν Κεφάλαια και βιβλίο. Σε κάθε περίπτωση, η συμπλήρωση
των  μεταδεδομένων  επιτρέπει  την  κατηγοριοποίηση του  παραγόμενου  όγκου
πληροφορίας και διευκολύνει τις αναζητήσεις των χρηστών.

Τα  μεταδεδομένα  θα  συμπληρωθούν  σε  ξεχωριστές  φόρμες  και  θα  παραδοθούν  από  τη
Συγγραφική Ομάδα κατά την υποβολή του τελικού υλικού.

Ειδικότερα, για τα μεταδεδομένα σε επίπεδο βιβλίου ορίζονται στον Πίνακα \ref{table:book_metadata}.

\begin{table} [ht] \centering
\caption{Μεταδεδομένα σε επίπεδο βιβλίου.}
\vspace{2mm}
\begin{tabular} {p{0.55\linewidth} p{0.15\linewidth} p{0.15\linewidth}}
\hline
	\textbf{Πεδίο}	&\textbf{Υποχρεωτικό}	&\textbf{Προαιρετικό}\tabularnewline
\hline
		&	    &	\tabularnewline
Τίτλος		& \centering $\surd$	    &\tabularnewline
Υπότιτλος	&	    	&\centering $\surd$	\tabularnewline
Εναλλακτικός τίτλος &  & \centering $\surd$	\tabularnewline
Συγγραφέας	&\centering $\surd$	    &	\tabularnewline
Συν-συγγραφείς	&\centering $\surd$	    &	\tabularnewline
Υπεύθυνος Κριτικής Ανάγνωσης / Επιμέλειας έκδοσης	&	    &\centering $\surd$	\tabularnewline
Γλωσσικός Επιμελητής	&	    &\centering $\surd$	\tabularnewline
Τεχνικοί Συντελεστές – Γραφιστική επιμέλεια	&	    &\centering $\surd$	\tabularnewline
Θεματική κατηγοριοποίηση (από ελεγχόμενο Κατάλογο
Θεματικών Όρων με τα γνωστικά αντικείμενα / επιστημονικές
εξειδικεύσεις ανά Θεματική Περιοχή / Θεματικό Πεδίο)	&\centering $\surd$	    &	\tabularnewline
Περίληψη	&\centering $\surd$	    &	\tabularnewline
Πίνακας περιεχομένων (σε μορφή κειμένου)	&\centering $\surd$	    &	\tabularnewline
Πληροφορίες σχετικά με τους Συγγραφείς	&\centering $\surd$	    &	\tabularnewline
	&	    &	\tabularnewline
\hline
\end{tabular}
\label{table:book_metadata}
\end{table}


Πέρα από τα γενικά μεταδεδομένα σε επίπεδο βιβλίου, θα πρέπει να καταχωριστούν και
συγκεκριμένα μεταδεδομένα σε επίπεδο Κεφαλαίου. Ειδικότερα, τα μεταδεδομένα σε επίπεδο Κεφαλαίου δίνονται στον πίνακα \ref{table:chapter_metadata}

\begin{table} [h!] \centering
\caption{Μεταδεδομένα σε επίπεδο Κεφαλαίου.}
\vspace{2mm}
\begin{tabular} {p{0.55\linewidth} p{0.15\linewidth} p{0.15\linewidth}}
\hline
	\textbf{Πεδίο}	&\textbf{Υποχρεωτικό}	&\textbf{Προαιρετικό}\tabularnewline
\hline
		&	    &	\tabularnewline
Τίτλος		& \centering $\surd$	    &\tabularnewline
Συγγραφείς	& \centering $\surd$	    	&	\tabularnewline
Βιβλίο, μέρος του οποίου αποτελεί το Κεφάλαιο & \centering $\surd$ & 	\tabularnewline
Θέμα – Θεματικές Περιοχές (από ελεγχόμενο Κατάλογο
Θεματικών Όρων με τα γνωστικά αντικείμενα /
επιστημονικές εξειδικεύσεις ανά Θεματική Περιοχή /
Θεματικό Πεδίο)	&\centering $\surd$	    &	\tabularnewline
Σύνοψη – περίληψη	&\centering $\surd$	    &	\tabularnewline
Πίνακας περιεχομένων (παραγράφων)	&\centering $\surd$	    &	\tabularnewline
Πίνακες μαθησιακών αντικειμένων (πινάκων, σχημάτων, ...)	&	    &\centering $\surd$	\tabularnewline
Προαπαιτούμενη γνώση	&	    &\centering $\surd$	\tabularnewline
Χρονική κάλυψη	&	    & \centering $\surd$	\tabularnewline
Γεωγραφική κάλυψη	&	    &\centering $\surd$	\tabularnewline
	&	    &	\tabularnewline
\hline
\end{tabular}
\label{table:chapter_metadata}
\end{table}

Τέλος, είναι απαραίτητο να συνοδεύονται από μεταδεδομένα τα σημαντικά σε περιεχόμενο
μαθησιακά αντικείμενα (πίνακες, εικόνες, διαδραστικά αντικείμενα κλπ.), τα οποία θα
καταχωρίζονται από τον δημιουργό κατά την υποβολή του συγγράμματος.Τα μεταδεδομένα
ανά κατηγορία περιεχομένου περιγράφονται στις παραγράφους του Κεφαλαίου \ref{chap:writing}
παρακάτω.
\section{Πνευματικά δικαιώματα και χρήση του υλικού}

Ο Συγγραφέας θα παραχωρεί στον ΣΕΑΒ την αποκλειστική Άδεια Εκμετάλλευσης του
έργου του, με την οποία επιτρέπει στον ΣΕΑΒ να ασκήσει ορισμένες από τις εξουσίες του
περιουσιακού δικαιώματος (βλ. Προσάρτημα IV – Όροι και Προϋποθέσεις Συμμετοχής
- Άδεια Εκμετάλλευσης), ωστόσο, αυτοδίκαια, θα διατηρεί το ηθικό δικαίωμα του
δημιουργού.
