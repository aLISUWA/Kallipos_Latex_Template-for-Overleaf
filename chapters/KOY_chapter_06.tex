\chapter{Συγγραφή με την κλάση kalliposstd }\label{chap:kallipos-std}
\begin{refsection}
\chapterauthor{Α. Συρόπουλος}
\chapterauthor{Γ. Παπανικολάου}
\begin{summary}{Σε αυτό το κεφάλαιο:}
Περιγράφονται οι εντολές και τα ειδικά περιβάλλοντα του τύπου εγγράφου βιβλίο kalliposstd. Δίνονται επίσης παραδείγματα χρήσης τους. \\
\textbf{Προαπαιτούμενη γνώση}:~Βασικές γνώσεις συγγραφής σε περιβάλλον~\XeLaTeX.
\end{summary}
\section{Λίγα λόγια για τον τύπο εγγράφου βιβλίο Kallipos standard}
Ο τύπος εγγράφου <<βιβλίο Kallipos standard>> (kalliposstd) δημιουργήθηκε από τον Α. Συρόπουλο με σκοπό να διευκολύνει τους συγγραφείς \XeLaTeX\ του Έργου ΚΑΛΛΙΠΟΣ+. Παρέχει τη δυνατότητα μορφοποίησης των βιβλίων με σχετικά ενιαίο τρόπο και απαλλάσσει τους συγγραφείς από την ανάγκη να δαπανήσουν επιπρόσθετο χρόνο ασχολούμενοι με τη μορφοποίηση του βιβλίου τους.

Ο τύπος εγγράφου kalliposstd είναι μια προέκταση του τυποποιημένου τύπου εγγράφου book που παρέχει κάθε διανομή \TeX~. Εισάγει έναν περιορισμένο αριθμό από ειδικά περιβάλλοντα και εντολές ώστε να διευκολυνθούν οι συγγραφείς \XeLaTeX\ κατά τη διαδικασία της συγγραφής.

\section{Προετοιμασία για συγγραφή}

\subsection{Εργασία στον τοπικό υπολογιστή}
Οι συγγραφείς \XeLaTeX\ συνήθως χρησιμοποιούν τα εργαλεία συγγραφής με τα οποία είναι εξοικειωμένοι. Αν
είστε νέος στο \XeLaTeX\ μπορείτε να δοκιμάσετε γνωστά εργαλεία συγγραφής, όπως τα προγράμματα TexMaker, TexWorks, ή απλώς κάποιο πρόγραμμα επεξεργασίας κειμένου όπως emacs, vim, gedit, notepad++ και πολλά άλλα. Αν είστε νέος στο \XeLaTeX\ μπορείτε εύκολα και απλά να ξεκινήσετε τη συγγραφή τροποποιώντας το αρχείο KOY\_master.tex και τα αρχεία του φακέλου \textbf{chapters}.

Στο μέλλον ο τύπος εγγράφου kalliposstd θα διατεθεί και σε online πλατφόρμες όπως οι ShareLatex και Overleaf.

\subsection{Καταφόρτωση του πρότυπου}
Μπορείτε να καταφορτώσετε τον kalliposstd με τη μορφή συμπιεσμένου αρχείου από τον ακόλουθο υπερσύνδεσμο:

\url{https://www.kallipos.gr/images/kalliposplus/A4/kalliposstd.zip}

Για να αποσυμπιέσετε το αρχείο χρησιμοποιήστε το γραφικό περιβάλλον (Windows και GNU/LINUX).

\subsubsection{Περιεχόμενα του αρχείου kalliposstd.zip}
Στον φάκελο \textbf{fonts} περιέχονται οι γραμματοσειρές που συνοδεύουν τον kalliposstd. Προκειμένου να μπορέσετε να εργαστείτε με τον kalliposstd θα χρειαστεί πρώτα να εγκαταστήσετε τις γραμματοσειρές όπως περιγράφεται παρακάτω.

Στον φάκελο \textbf{chapters} περιέχονται τα κεφάλαια του Οδηγού για τους συγγραφείς \XeLaTeX\, ενώ το βασικό αρχείο (master file) του οδηγού βρίσκεται στον αρχικό φάκελο (KOY\_master.tex).

Στον φάκελο \textbf{images} τοποθετούνται τα αρχεία εικόνων που χρησιμοποιείτε στο βιβλίο σας.

Στον φάκελο \textbf{pdf} περιέχεται ο οδηγός σε μορφή pdf. Στον φάκελο pdf τοποθετείτε τα αρχεία pdf που παράγετε, τόσο το πλήρες βιβλίο, όσο και τα επιμέρους κεφάλαια και παραρτήματα σε ξεχωριστά αρχεία pdf.

Σημειώνεται ότι η δομή του φακέλου διευκολύνει τη διαδικασία παράδοσης του υλικού, καθώς το μόνο που χρειάζεται να κάνουν οι συγγραφείς  \XeLaTeX\ είναι να αντιγράψουν τα περιεχόμενα του φακέλου στον φάκελο \textbf{02\_tex} (βλέπε σελ. \pageref{package}).

Tο αρχείο \textbf{kalliposstd.cls} ορίζει τον τύπο εγγράφου kalliposstd. Στην
προκαθορισμένη λειτουργία του τύπου εγγράφου Kalliposstd θεωρείται ότι χρησιμοποιούμε τον τύπο
εγγράφου \texttt{book} με τις επιλογές \texttt{twoside} και \texttt{12pt}.

O χρήστης του αρχείου \texttt{kalliposstd.cls}, θα πρέπει να χρησιμοποιεί τη μηχανή \XeLaTeX.
Έτσι, η πρώτη γραμμή του αρχείου του θα έχει την παρακατω μορφή:
\begin{center}
\verb=\documentclass{kalliposstd}=
\end{center}
ή
\begin{center}
\verb=\documentclass[draft]{kalliposstd}=
\end{center}

\subsection{Εγκατάσταση γραμματοσειρών}
\subsubsection{Λειτουργικό σύστημα GNU/LINUX - Συστήματα τύπου UNIX }

Οι γραμματοσειρές διαμοιράζονται μαζί με το πρότυπο στα συμπιεσμένα αρχεία arno-pro.zip και arimo.zip. Για να εγκατασταθούν αποσυμπιέστε τα αρχεία από το γραφικό περιβάλλον ή με τη βοήθεια της εντολής:
\begin{verbatim}
unzip arno-pro.zip
\end{verbatim}
Στη συνέχεια μπορείτε να εγκαταστήσετε τις γραμματοσειρές με έναν από τους παρακάτω τρεις τρόπους:
\begin{enumerate}
\item Αντιγράψτε τον αποσυμπιεσμένο φάκελο Arno Pro (αρχεία otf) στον φάκελο /usr/share/fonts/opentype και τον φάκελο arimo στον φάκελο /usr/share/fonts/truetype. Απαραίτητο είναι να διαθέτετε δικαιώματα διαχειριστή:
\begin{verbatim}
sudo cp -R $HOME/path_to_folder/arno-pro /usr/share/fonts/opentype/
sudo cp -R $HOME/path_to_folder/arimo /usr/share/fonts/truetype/
\end{verbatim}
αντικαθιστώντας το \verb=/home/username/path_to_folder/= με το μονοπάτι για τον αποσυμπιεσμένο φάκελο. Αφού πραγματοποιήσετε την αντιγραφή των αρχείων εκτελέστε την εντολή:
\begin{verbatim} sudo fc-cache -f -v
\end{verbatim}
\item Εναλλακτικά, μπορείτε να εγκαταστήσετε τις γραμματοσειρές για το περιβάλλον του χρήστη σας αντιγράφοντάς τες στον φάκελο /home/.fonts. Αν ο φάκελος δεν υπάρχει, μπορούμε να τον δημιουργήσουμε.
\item Τέλος, με διπλό κλικ πάνω στα αρχεία των γραμματοσειρών ανοίγει η εφαρμογή \emph{Font Viewer} που μας επιτρέπει την εγκατάστασή τους από το γραφικό περιβάλλον.
\end{enumerate}
Για την αναπαραγωγή του παρόντος Οδηγού θα χρειαστεί να εγκαταστήσετε με τον ίδιο τρόπο τη γραμματοσειρά Consolas που μπορείτε να καταφορτώσετε από τη διεύθυνση \url{https://www.freefonts.io/downloads/consolas/}
\subsubsection{Λειτουργικό σύστημα WINDOWS 10}
Για εγκατάσταση των γραμματοσειρών στο λειτουργικό σύστημα WINDOWS 10:
\begin{enumerate}
\item Ανοίξτε το control panel και στη συνέχεια ακολουθήστε τη διαδρομή \textbf{Control Panel} > \textbf{Appearance and Personalization} > \textbf{Fonts} (ή στα ελληνικά, \textbf{Πίνακας ελέγχου} > \textbf{Εμφάνιση και εξατομίκευση} > \textbf{Γραμματοσειρές})
\item Αντιγράψτε με Drag and Drop όλα τα αρχεία γραμματοσειρών στο παράθυρο του Πίνακα Ελέγχου που ανοίχτηκε στο προηγούμενο βήμα.
\end{enumerate}
\section{Διαμόρφωση του βασικού αρχείου (master file) - Δομή του φακέλου εργασίας}
Για δική σας διευκόλυνση είναι προτιμότερο να διαμορφώσετε το βασικό αρχείο .tex (master file) του βιβλίου
χρησιμοποιώντας σαν αφετηρία το master file που διαμοιράζεται με τον kalliposstd, τροποποιώντας το αρχείο
με όνομα KOY\_master.tex.

Αν χρησιμοποιείτε κάποια εξειδικευμένα πακέτα που δεν περιλαμβάνονται στο προοίμιο του βασικού αρχείου
μπορείτε να τα εισάγετε, ακολουθώντας πάντα την σωστή σειρά σύμφωνα με τις οδηγίες χρήσης τους.
Πριν εισάγετε πρόσθετα πακέτα βεβαιωθείτε ότι σας είναι απολύτως αναγκαία. Λάβετε επίσης υπόψιν σας ότι
ορισμένα πακέτα έχουν ήδη φορτωθεί από την kalliposstd.cls και δεν θα πρέπει να φορτωθούν για δεύτερη φορά.
Τα πακέτα αυτά είναι τα ακόλουθα:
\begin{multicols}{2}
\begin{itemize}
\item xltxtra
\item xgreek
\item ucharclasses
\item geometry
\item titlesec
\item suffix
\item tikz
\item enumitem
\item newfile
\item pdfpages
\item hyperxmp
\item hyperref
\end{itemize}
\end{multicols}
Θα πρέπει να διατηρήσετε την δομή του φακέλου εργασίας που σας διαμοιράστηκε, αφαιρώντας φυσικά τον φάκελο
που περιέχει τις γραμματοσειρές εργασίας. Η δομή αυτή θα ακολουθηθεί και κατά την παράδοση του πρωτογενούς
υλικού όπως εξηγείται παρακάτω. Επίσης θα πρέπει να ακολουθηθεί και ο τρόπος ονοματολογίας των αρχείων,
όπως έχει εξηγηθεί στο κεφάλαιο \ref{chap:best-practice}. Η αυστηρή τήρηση της δομής θα βοηθήσει τυχόν
εργασίες που θα χρειαστεί να γίνουν από τους συνεργάτες τεχνικής/γραφιστικής επεξεργασίας της ΚΟΥ.

Το βασικό αρχείο περιέχει διευκρινιστικά σχόλια που θα σας βοηθήσουν κατά τη διαδικασία της συγγραφής.

Αν έχετε ήδη προχωρήσει τη συγγραφή σας, δοκιμάστε να διαμορφώσετε την εργασία σας εισάγοντας τα επί μέρους
κεφάλαια στο βασικό αρχείο που σας δίνεται (τροποποίηση του ΚΟΥ\_master.tex).
\section{Τα πρώτα βήματα: μέγεθος σελίδας, γραμματοσειρές και οι πρώτες σελίδες του βιβλίου}
\subsection{Μέγεθος σελίδας}
Ο τύπος εγγράφου Kalliposstd παρέχει τις παρακάτω επιλογές:
\begin{description}
\item[\texttt{a4book}] Δημιουργεί ένα αρχείο PDF με διαστάσεις χαρτιού $205\,\text{mm}\times 290\,\text{mm}$.
Αυτή είναι η προκαθορισμένη επιλογή.
%
\item[\texttt{custombook}] Δημιουργεί ένα αρχείο PDF με διαστάσεις χαρτιού $170\,\text{mm}\times 240\,\text{mm}$,
%
\item[\texttt{draft}] Εμφανίζει ένα μαύρο πλαίσιο στην άκρη κάθε αράδας που είναι μεγαλύτερη από το
προκαθορισμένο μήκος,
%
\item[\texttt{nonderivative}] Επιλέγει την άδεια του εγγράφου. Στην περίπτωση αυτή είναι η «Creative Commons Αναφορά Δημιουργού - Μη Εμπορική Χρήση – Όχι Παράγωγα Έργα 4.0». Αν δεν δηλωθεί κάποια επιλογή χρησιμοποιείται η προεπιλεγμένη άδεια που είναι η «Creative Commons Αναφορά Δημιουργού - Μη Εμπορική Χρήση - Παρόμοια Διανομή 4.0».
\end{description}

Για την παράδοση του βιβλίου στο πλαίσιο του Έργου ΚΑΛΛΙΠΟΣ+ θα χρειαστεί να χρησιμοποιηθεί η διάσταση 205 mm × 290 mm (προεπιλεγμένη - \texttt{a4book}). Η επιλογή \texttt{custombook} μπορεί να χρησιμοποιηθεί, εάν χρειαστεί, για την εκτύπωση του βιβλίου σε μικρότερη διάσταση. Για παράδειγμα:
\begin{center}
\verb|\documentclass[custombook]{kalliposstd}|
\end{center}
\subsection{Επιλογή γραμματοσειράς}
Ο τύπος εγγράφου kalliposstd διαμοιράζεται με τις γραμματοσειρές Arimo (γραμματοσειρά sans serif) και Arno Pro (γραμματοσειρά serif). Οι συγκεκριμένες γραμματοσειρές διατίθενται με ανοικτές άδειες γεγονός που επιτρέπει τον νόμιμο διαμοιρασμό και χρήση τους. Υποστηρίζουν όλες τις γλυφές που χρησιμοποιούνται από τον kalliposstd και επιτρέπουν τη συγγραφή σε ελληνικό πολυτονικό και κυριλλικό αλφάβητο.

Για λόγους ομοιομορφίας προτείνουμε τη χρήση των γραμματοσειρών Arno Pro (γραμματοσειρά serif) και Arimo
(γραμματοσειρά sans serif). Ωστόσο οι συγγραφείς μπορούν να επιλέξουν τη γραμματοσειρά που επιθυμούν αρκεί να
έχουν άδεια για τη χρήση της.
Ενδεικτικά, γραμματοσειρές που μπορούν να χρησιμοποιηθούν είναι οι Minion Pro, Times
New  Roman, κλπ., 11--12\,pt (σε κυρίως κείμενο), 10--11 pt (σε υποσημειώσεις, λεζάντες).

Οι συγγραφείς θα πρέπει να τις σημειώσουν στην αρχή του βασικού αρχείου (master file), όπως για παράδειγμα
παρακάτω:
\begin{verbatim}
       \documentclass{kalliposstd}
        ...
       \begin{document}
       \setmainfont[Mapping=tex-text,Ligatures=Common]{Arno Pro}
       \setsansfont[Scale=MatchLowercase,Mapping=tex-text]{Arimo}
\end{verbatim}
\subsection{Συλλαβισμός - γλωσσική υποστήριξη}
Όταν σημειώνουμε λατινικούς χαρακτήρες, ενεργοποιούνται αυτόματα οι κανόνες συλλαβισμού της αγγλικής γλώσσας,
ενώ όταν σημειώνουμε ελληνικούς χαρακτήρες, ενεργοποιούνται οι κανόνες συλλαβισμού της
ελληνικής γλώσσας. Αν θέλουμε να ενεργοποιούνται οι κανόνες συλλαβισμού της γερμανικής θα πρέπει
να προσθέσουμε την εντολή:
\begin{center}
\verb=\setTransitionsForLatin{\setlanguage{german}}{}=
\end{center}
στο προοίμιο του κώδικα \LaTeX. Άλλες δυνατές τιμές είναι η \texttt{french} (για γαλλικά),
\texttt{italian} (για ιταλικά) κ.λπ.
Αν όμως θέλουμε να ενεργοποιήσουμε τους κανόνες συλλαβισμού της ρωσικής, όταν το \XeLaTeX\
συναντά κυριλλικό κείμενο, τότε θα πρέπει να χρησιμοποιήσουμε την παρακάτω εντολή:
\begin{center}
\verb=\setTransitionsForCyrillics{\setlanguage{russian}}{}=
\end{center}
Αν θέλουμε να γράψουμε αρχαίο κείμενο, τότε, για να γίνει σωστός συλλαβισμός θα πρέπει να
ενεργοποιήσουμε τους ανάλογους κανόνες συλλαβισμού:
\begin{verbatim}
{\setlanguage{ancientgreek} Φανερὸν δὲ ἐκ τῶν εἰρημένων ὅτι
 ἀνάγκη περὶ τούτων ἔχειν πρῶτον τὰς προτάσεις· τὰ γὰρ τεκμήρια
 καὶ τὰ εἰκότα καὶ τὰ σημεῖα προτάσεις εἰσὶν ῥητορικαί· ὅλως μὲν
 γὰρ συλλογισμὸς ἐκ προτάσεών ἐστιν, τὸ δ᾽ ἐνθύμημα συλλογισμός
 ἐστι συνεστηκὼς ἐκ τῶν εἰρημένων προτάσεων.}
\end{verbatim}
Προσέξτε ότι το αρχαίο κείμενο γράφεται ανάμεσα σε άγκιστρα και η εντολή \verb=\setlanguage{ancientgreek}=
ενεργοποιεί {\em τοπικά} τους σχετικούς κανόνες συλλαβισμού.

\subsection[Τίτλος συγγράμματος, κύριος συγγραφέας, συν-συγγραφείς,
\string\hfill\string\break\space συντελεστές και αφιέρωση ]{Τίτλος συγγράμματος, κύριος συγγραφέας, συν-συγγραφείς, συντελεστές και αφιέρωση}
Ο τίτλος συγγράμματος και ο κύριος συγγραφέας θα πρέπει να δηλωθούν στην αρχή του εγγράφου γιατί  είναι
απαραίτητα και χρησιμοποιούνται στη στοιχειοθεσία του ποδιού της πρώτης σελίδας του κάθε κεφαλαίου που
περιλαμβάνει  τον τίτλο του συγγράμματος, τον κύριο συγγραφέα και την άδεια Creative Commons με την οποία διατίθεται το σύγγραμμα. Οι επισημάνσεις αυτές τοποθετούνται στην αρχική σελίδα κάθε κεφαλαίου γιατί πολύ συχνά τα κεφάλαια των συγγραμμάτων του ΚΑΛΛΙΠΟΥ καταφορτώνονται και διακινούνται σαν ξεχωριστά αρχεία. Για να δηλώσετε το όνομα του κύριου συγγραφέα και τον τίτλο του συγγράμματος χρησιμοποιείτε τις εντολές \verb|\primaryauthor| και \verb|\PDFtitle|. Στον παρόντα Οδηγό η σύνταξη έχει ως εξής:
\begin{verbatim}
\begin{document}
\primaryauthor{Κεντρική Ομάδα Υποστήριξης}
\PDFtitle{Οδηγίες για συγγραφείς XeLaTeX}
\end{verbatim}

Στη συνέχεια θα χρειαστεί να συμπληρώσετε έναν βραχύ τίτλο του βιβλίου σας που θα τοποθετηθεί στην κεφαλίδα
των μονών σελίδων των κεφαλαίων. Αυτό γίνεται με την \verb|\PDFshorttitle|. Για παράδειγμα:
\begin{verbatim}
\PDFshorttitle{Οδηγίες για συγγραφείς \XeLaTeX}
\end{verbatim}

Με την εντολή \verb|\PDFauthor| εισάγετε τα ονόματα όλων των συγγραφέων του βιβλίου, ξεκινώντας από τον κύριο συγγραφέα και συνεχίζοντας με τους υπόλοιπους συγγραφείς με αλφαβητική σειρά. Σημειώνετε το επώνυμο του συγγραφέα και το πρώτο γράμμα του μικρού του ονόματος ακολουθούμενο από τελεία. Τα ονόματα των συγγραφέων διαχωρίζονται μεταξύ τους με κόμμα. Για παράδειγμα:
\begin{verbatim}
\PDFauthor{Παπανικολάου Γ., Συρόπουλος Α. και Τελευταίος_Συγγραφέας Α.}
\end{verbatim}
Τα παραπάνω στοιχεία χρησιμοποιούνται για την παραγωγή της βιβλιογραφικής αναφοράς του βιβλίου που βρίσκεται
στην τέταρτη σελίδα και στα μεταδεδομένα του αρχείου PDF.

Τέλος, με την εντολή \verb|\PDFyear| δηλώνετε το έτος συγγραφής του βιβλίου, το οποίο χρησιμοποιείται για την παραγωγή της νομικής σημείωσης της τέταρτης σελίδας, του ποδιού της πρώτης σελίδας κάθε κεφαλαίου και στα μεταδεδομένα του αρχείου PDF.
\begin{verbatim}
\PDFyear{2021}
\end{verbatim}

Σημειώνεται ότι δεν χρειάζεται να συμπληρώσετε το όρισμα της \verb|PDFdoi|. Αυτό θα γίνει από τα μέλη
της ΚΟΥ. Στο τέλος το βασικό σας αρχείο (master file) θα μοιάζει καπως έτσι:
\begin{verbatim}
\primaryauthor{Κεντρική Ομάδα Υποστήριξης}
\PDFtitle{Οδηγίες για συγγραφείς XeLaTeX}
\PDFshorttitle{Οδηγίες για συγγραφείς \XeLaTeX}
\PDFdoi{}                                       %Συμπληρώνεται από την ΚΟΥ (post-production)
\PDFauthor{Κεντρική Ομάδα Υποστήριξης}
\PDFyear{2021}
\end{verbatim}

Για τη διαμόρφωση της πρώτης σελίδας που περιλαμβάνει μόνο τον τίτλο του βιβλίου χρησιμοποιούμε την εντολή
\verb|\soletitlepage{τίτλος βιβλίου}|.

Στην τρίτη σελίδα του βιβλίου επαναλαμβάνεται ο τίτλος του βιβλίου και αναγράφονται τα μέλη της συγγραφικής ομάδας. Αυτό γίνεται χρησιμοποιώντας το περιβάλλον authorpage. Αυτό δέχεται δύο ορίσματα: ένα υποχρεωτικό (τον τίτλο του βιβλίου) και ένα κατ’ επιλογήν όρισμα (υπότιτλος βιβλίου). Για παράδειγμα:
\begin{center}
\verb|\begin{authorpage}{τίτλος}[υπότιτλος]|
\end{center}
Στο ίδιο περιβάλλον θα πρέπει να σημειώσουμε τους συγγραφείς του βιβλίου ξεκινώντας με τον κύριο συγγραφέα
και συνεχίζοντας με τους υπόλοιπους συγγραφείς αλφαβητικά. Γι' αυτόν τον σκοπό  χρησιμοποιούμε το περιβάλλον
\emph{authors}:
\begin{center}
\begin{tabular}{l}
\verb=\begin{authors}=\\
\verb=Απόστολος Συρόπουλος\\ Οδός Σοφίας 3\\ Ξάνθη \and=\\
\verb=Γεώργιος Παπανικολάου\\ Οδός Ελπίδος 3\\ Καβάλα=\\
\verb=\end{authors}=\\
\verb=\end{authorpage}=
\end{tabular}
\end{center}
Παρατηρήστε ότι χωρίζουμε τα διάφορα τμήματα που αφορούν έναν συγγραφέα με δύο αντιπλάγιες (\verb=\\=) και ξεκινάμε την εισαγωγή των στοιχείων επόμενου συγγραφέα σημειώνοντας την εντολή \verb|and|.

Αν θέλουμε να προσθέσουμε μια αφιέρωση στο βιβλίο, μπορούμε να χρησιμοποιήσουμε το περιβάλλον \texttt{dedication} πριν δώσουμε την εντολή \verb|\mainmatter| . Παρακάτω δίνουμε ένα απλό παράδειγμα χρήσης του περιβάλλοντος \texttt{dedication}:
\begin{center}
\begin{tabular}{l}
\verb=\begin{dedication}=\\
\verb=Αφιερώνεται\\ =\\
\verb=στους συγγραφείς \XeLaTeX,\\ =\\
\verb=του Έργου ΚΑΛΛΙΠΟΣ+\\ =\\
\verb=ΚΟΥ \\[6pt]=\\
\verb=... =\\
\verb=\end{dedication} =\\
\end{tabular}
\end{center}
Προσέξτε ότι βάλαμε την εντολή \verb|[6pt]| απλώς για να ξεχωρίσει η μία αφιέρωση από την επόμενη, την οποία δεν σημειώσαμε για λόγους συντομίας.

Συνοψίζοντας, για παράδειγμα, παρατίθεται ο κώδικας των αρχικών σελίδων του παρόντος Οδηγού:
\begin{center}
\begin{tabular}{l}
\verb=\primaryauthor{Κεντρική Ομάδα Υποστήριξης}=\\
\verb=\booktitle{Οδηγίες για συγγραφείς \XeLaTeX}=\\
\verb=%------- Αρχικές σελίδες  -----------------------------------=\\
\verb=\soletitlepage{Οδηγίες για συγγραφείς \XeLaTeX}=\\
\verb=\begin{authorpage}{Οδηγίες για συγγραφείς \XeLaTeX}[Έκδοση 1η]=\\
\verb=\begin{authors}=\\
\verb=Κεντρική Ομάδα Υποστήριξης\\ Κάλλιπος, Ανοικτές Ακαδημαϊκές Εκδόσεις\\ Αθήνα =\\
\verb=%\and Όνομα 2ου συγγραφέα\\ Ιδιότητα\\ Ίδρυμα=\\
\verb=\end{authors}=\\
\verb=\end{authorpage}=\\
\verb=\copyrightpage{Σταματίνα Κουτσιλέου}{Αλεξάνδρα Θεοδωράκη}=\\
\verb={Γιώργος Παπανικολάου}{12345}=\\
\verb=\begin{dedication}=\\
\verb=Αφιερώνεται\\ =\\
\verb=στους συγγραφείς \XeLaTeX,\\ =\\
\verb=του Έργου ΚΑΛΛΙΠΟΣ+\\ =\\
\verb=ΚΟΥ \\[6pt]=\\
\verb=\end{dedication} =\\
\verb= %----------------------------------------------------------=\\
\end{tabular}
\end{center}

\subsection{Επιλογή της άδειας Creative Commons}
Τα συγγράμματα του Κάλλιπου διατίθενται με άδειες Creative Commons. Οι χρησιμοποιούμενες άδειες είναι
οι:
\begin{enumerate}
\item Creative Commons Αναφορά Δημιουργού - Μη Εμπορική Χρήση - Παρόμοια Διανομή 4.0. και,
\item Creative Commons Αναφορά Δημιουργού - Μη Εμπορική Χρήση - Όχι Παράγωγα Έργα 4.0.
\end{enumerate}

Από προεπιλογή η κλάση kalliposstd χρησιμοποιεί την άδεια Creative Commons Αναφορά Δημιουργού -
Μη Εμπορική Χρήση - Παρόμοια Διανομή 4.0. για την παραγωγή της νομικής σημείωσης που βρίσκεται
στην 4η σελίδα
του βιβλίου, καθώς αυτή είναι η άδεια την οποία χρησιμοποιεί η πλειονότητα των συγγραφέων. Για
να τροποποιήσετε την προεπιλογή της άδειας σε Creative Commons Αναφορά Δημιουργού - Μη Εμπορική
Χρήση - Όχι Παράγωγα Έργα 4.0. θα πρέπει να δηλώσετε την επιλογή \emph{nonderivative} στην αρχή
του εγγράφου ως εξής:
\begin{center}
 \verb|\documentclass[nonderivative]{kalliposstd}|
\end{center}

\subsection{Πρόλογος, εισαγωγή και κεφάλαια χωρίς αρίθμηση}
Στις αρχικές σελίδες του βιβλίου μπορεί να περιλαμβάνονται πρόλογος, εισαγωγή και άλλα στοιχεία
 χωρίς αρίθμηση. Η εντολή \verb|\schapter| μπορεί να χρησιμοποιηθεί για κεφάλαια χωρίς αρίθμηση
 (τέτοια μπορούν να εισαχθούν πριν ή μετά τους διάφορους πίνακες περιεχομένων). O τίτλος του
 κεφαλαίου που δημιουργείται με αυτήν την εντολή θα εμφανιστεί στον πίνακα περιεχομένων του
 βιβλίου. Οι αρχικές σελίδες αριθμούνται με λατινικούς χαρακτήρες, ενώ δεν λαμβάνουν αρίθμηση ως
 κεφάλαια. Τα κεφάλαια αυτά ακολουθούν την εντολή \verb|\frontmatter| που πρέπει πάντα να
 τοποθετείται μετά τις εντολές που ακολουθούν τη σελίδα του copyright και της αφιέρωσης.

Το όνομα του συγγραφέα του προλόγου ή της εισαγωγής τοποθετείται στην αρχή με τη βοήθεια της
εντολής \verb|\chapterauthor|. Η ίδια εντολή χρησιμοποιείται και για το όνομα του συγγραφέα στην
αρχή κάθε κεφαλαίου του κύριου μέρους του βιβλίου. Η εντολή αυτή δεν χρειάζεται να χρησιμοποιηθεί
σε μονοσυγγραφικά συγγράμματα του Έργου, παρά μόνο στις περιπτώσεις πολυσυγγραφικών συγγραμμάτων.
Η εντολή μπορεί να δεχτεί έως δύο ορίσματα, εκ των οποίων το δεύτερο είναι προαιρετικό. Το δεύτερο,
προαιρετικό όρισμα υφίσταται αν θέλουμε να βάλουμε επιπλέον πληροφορίες κάτω από τον τίτλο του
κεφαλαίου αλλά όχι στα περιεχόμενα. Το δεύτερο προαιρετικό όρισμα πρέπει να σημειωθεί σε αγκύλες,
όπως φαίνεται παρακάτω:
Παράδειγμα:
\begin{verbatim}
\schapter{Εισαγωγή}
\chapterauthor{Κεντρική Ομάδα Υποστήριξης}
               [Κεντρική Ομάδα Υποστήριξης\\
               ΚΑΛΛΙΠΟΣ, Ανοικτές Ακαδημαϊκές Εκδόσεις\\
               \url{https://helpdesk.kallipos.gr}]
\end{verbatim}
Στο παραπάνω παράδειγμα στον πίνακα περιεχομένων θα φαίνεται μόνο το όνομα του συγγραφέα (με
αντιπλάγιες θα μπορούσαν να προστεθούν περισσότερα στοιχεία, ωστόσο κάτι τέτοιο θα επιβάρυνε
ιδιαίτερα τον πίνακα περιεχομένων). Στην αρχή του αντίστοιχου κεφαλαίου θα φαίνονται σε τρεις
γραμμές αντίστοιχα τα παρακάτω: Κεντρική Ομάδα Υποστήριξης, ΚΑΛΛΙΠΟΣ, Ανοικτές Ακαδημαϊκές Εκδόσεις και
ο υπερσύνδεσμος  \url{https://helpdesk.kallipos.gr}.
Αν το σύγγραμμα δεν είναι πολυσυγγραφικό, η τοποθέτηση του ονόματος παραλείπεται και η εντολή
\verb|\chapterauthor| δεν χρειάζεται να χρησιμοποιηθεί.

\subsection{Εισαγωγή των μεταδεδομένων του βιβλίου}

Για να εισάγετε τα μεταδεδομένα του βιβλίου θα χρειαστεί να επεξεργαστείτε το αρχείο \texttt{metadata.tex} 
που βρίσκεται στον φάκελο \textbf{chapters}. Το αρχείο περιέχει σχόλια που σας κατευθύνουν ώστε να
συμπληρώσετε μόνο τα μεταδεδομένα που συμπληρώνει ο συγγραφέας όπως τίτλος και υπότιτλος, λέξεις κλειδιά κλπ. Κατά την εισαγωγή των μεταδεδομένων θα πρέπει να χρησιμοποιούνται αυστηρά χαρακτήρες unicode και δεν θα πρέπει να τοποθετούνται εντολές \LaTeX.

Παρατηρήστε ότι στο ίδιο αρχείο, αν χρειαστεί, γίνεται και η ρύθμιση της εμφάνισης των υπερσυνδέσμων.
Η περιοχή που θα χρειαστεί να επεξεργαστούν οι συγγραφείς είναι η ακόλουθη:
\begin{verbatim}
%-------------------------------------------------------------
%			ΜΕΤΑΔΕΔΟΜΕΝΑ ΠΟΥ ΣΥΜΠΛΗΡΩΝΟΝΤΑΙ ΑΠΟ ΤΟΝ ΣΥΓΓΡΑΦΕΑ
%-------------------------------------------------------------
    	pdftitle={Οδηγίες για συγγραφείς XeLaTeX},
	pdfsubtitle={Ένας σύντομος Οδηγός},
	pdfauthor={Συγγραφείς με επώνυμο πρώτο γράμμα ονόματος χωρισμένοι με κόμμα},
	pdfsubject={Αντικείμενο του συγγράμματος πχ. υπολογιστικά δίκτυα},
	pdfkeywords={Λέξεις κλειδιά χωρισμένες με κόμμα},
	pdflicenseurl={https://creativecommons.org/licenses/by-nc-sa/4.0/legalcode.el},
% Εναλλακτικά, για βιβλία με άδεια cc-non-derivative να χρησιμοποιηθεί
% ο ακόλουθος σύνδεσμος αντί για τον παραπάνω
% https://creativecommons.org/licenses/by-nc-nd/4.0/legalcode.el
%
\end{verbatim}
Προσέξτε ιδιαίτερα την εισαγωγή της ορθής άδειας στα μεταδεδομένα. Από προεπιλογή έχει τοποθετηθεί ο υπερσύνδεσμος
της άδειας Creative Commons Αναφορά Δημιουργού - Μη Εμπορική Χρήση - Παρόμοια Διανομή 4.0. Αν το 
περιεχόμενό σας αδειοδοτείται με Creative Commons Αναφορά Δημιουργού - Μη Εμπορική Χρήση - Όχι Παράγωγα Έργα 4.0.
θα χρειαστεί να εισαγάγετε τον σύνδεσμο που σας δίνεται στο αρχείο.
\section{Το κύριο μέρος του βιβλίου}
Μόλις ξεκινήσει το κύριο μέρος του βιβλίου, θα πρέπει να χρησιμοποιήσουμε την εντολή
\verb|\mainmatter| για να αλλάξει η αρίθμηση των σελίδων και να ενεργοποιηθεί το σχετικό στυλ
σελίδας. Όλες οι εντολές τμηματοποίησης ( \verb|\chapter| , \verb|\section| κ.λπ.) λειτουργούν όπως γνωρίζουμε, αλλά το αποτέλεσμα είναι διαφορετικό από ό,τι στον τύπο εγγράφου texttt{book}.
\subsection{Η δομή των κεφαλαίων}
Αμέσως μετά τον τίτλο και τον συγγραφέα ή τους συγγραφείς ενός κεφαλαίου μπορούμε να προσθέσουμε
μια περίληψη, μια περιγραφή, τους μαθησιακούς στόχους κοκ. του κεφαλαίου. Στη συνέχεια,
αναπτύσσονται τα διάφορα τμήματα του κεφαλαίου. Στο τέλος του κεφαλαίου συμπεριλαμβάνονται σε
ξεχωριστό τμήμα οι ασκήσεις, τα προβλήματα ή κριτήρια αξιολόγησης και η βιβλιογραφία. Η βιβλιογραφία
τοποθετείται στο τέλος κάθε κεφαλαίου.
\subsubsection{Περίληψη}
Αμέσως μετά τον τίτλο και τον συγγραφέα ή τους συγγραφείς ενός κεφαλαίου μπορούμε να προσθέσουμε
μια περίληψη, μια περιγραφή, τους μαθησιακούς στόχους κοκ.
Αυτό μπορούμε να το προσθέσουμε με το περιβάλλον \texttt{summary}. Το περιβάλλον \texttt{summary} έχει ένα
υποχρεωτικό όρισμα που είναι ο τίτλος του κειμένου (Πχ. περίληψη, μαθησιακοί στόχοι κ.λπ.).
Κάτι που πρέπει να προσεχτεί ιδιαίτερα είναι το να μην αφήνετε κενή γραμμή αμέσως μετά την πρώτη
γραμμή του περιβάλλοντος. Δηλαδή, \textbf{ΔΕΝ} θα πρέπει να σημειώνετε το περιβάλλον όπως
παρακάτω:
\begin{center}
\begin{tabular}{l}
\verb=\begin{summary}{Σε αυτό το κεφάλαιο:}=\\
\\
\verb=Πριν από λίγες ημέρες ολοκληρώθηκε μια ενδιαφέρουσα...=\\
\verb=\end{summary}=\\
\end{tabular}
\end{center}
αλλά παραλείποντας την κενή γραμμή όπως φαίνεται παρακάτω:
\begin{center}
\begin{tabular}{l}
\verb=\begin{summary}{Σε αυτό το κεφάλαιο:}=\\
\verb=Πριν από λίγες ημέρες ολοκληρώθηκε μια ενδιαφέρουσα...=\\
\verb=\end{summary}=\\
\end{tabular}
\end{center}
\subsubsection{Ερωτήσεις, ασκήσεις, κριτήρια αξιολόγησης}
Αν θέλουμε να προσθέσουμε ασκήσεις, ερωτήσεις κ.λπ. στο βιβλίο ή σε κάποιο κεφάλαιο ή σε κάθε
κεφάλαιο, μπορούμε να χρησιμοποιήσουμε το περιβάλλον \texttt{exercises}. Αυτό το περιβάλλον μπορεί
να δεχτεί ένα κατ’ επιλογή όρισμα το οποίο αντιστοιχεί στο όνομα της ενότητας (πχ. «Ερωτήσεις»).
Αν δεν το σημειώσουμε, τότε θεωρείται ότι είναι η λέξη «Ασκήσεις». Το περιβάλλον χρησιμοποιεί μια
εκδοχή του περιβάλλοντος \texttt{enumerate} για την παράθεση των προβλημάτων.

Στην περίπτωση που \textbf{ΔΕΝ συμπεριλαμβάνουμε} τις απαντήσεις στα ερωτήματα ή τις ασκήσεις,
η σύνταξη για να τοποθετήσουμε τις ασκήσεις/ερωτήσεις μας στο τέλος του κεφαλαίου θα είναι όπως
παρακάτω:
\begin{center}
\begin{tabular}{l}
\verb=\section{Προβλήματα-Ερωτήσεις}=\\
\verb=\begin{exercises}[Προβλήματα]=\\
\verb=\item 1ο πρόβλημα=\\
\verb=\item 2ο πρόβλημα=\\
\verb=. . . . . . . . .=\\
\verb=\end{exercises}=
\end{tabular}
\end{center}
Στην περίπτωση αυτή θα δημιουργηθεί μια αριθμημένη λίστα με τίτλο <<Προβλήματα>>. Αν στο ίδιο κεφάλαιο
εκτός από προβλήματα έχουμε και ερωτήσεις, μπορούμε να τις προσθέσουμε στη συνέχεια χρησιμοποιώντας
εκ νέου το περιβάλλον \texttt{exercises} με τον τίτλο <<Ερωτήσεις>> ως εξής:
\begin{center}
\begin{tabular}{l}
\verb=\begin{exercises}[Ερωτήσεις]=\\
\verb=\item 1ο ερώτημα=\\
\verb=\item 2ο ερώτημα=\\
\verb=. . . . . . . . .=\\
\verb=\end{exercises}=
\end{tabular}
\end{center}
Αν εκτός από τις ερωτήσεις/ασκήσεις \textbf{θέλουμε να συμπεριλάβουμε και τις απαντήσεις} των
ερωτήσεων ή τις λύσεις των ασκήσεων τότε θα πρέπει να ακολουθήσουμε τα παρακάτω βήματα:
\begin{enumerate}
\item Ορίζουμε ένα νέο είδος απαντήσεων: \verb|newanswer{solutions}|. Το όρισμα της \verb|\newanswer|
μπορεί να είναι όποια ονομασία επιλέξουμε πχ. \texttt{solutionsA} ή ακόμα και μια ελληνική λέξη όπως \texttt{λύσεις}
στην περίπτωση όμως που επιλεγεί λέξη με ελληνικούς χαρακτήρες θα πρέπει να δώσουμε ιδιαίτερη
προσοχή στην πιστή αναγραφή της, τους τόνους κ.λπ.
\item Στη συνέχεια χρησιμοποιούμε το περιβάλλον \texttt{write\textit{solutions}} (προσοχή! το όνομα προκύπτει από το write + το όνομα που δώσαμε, δηλ. write+solutions και δεν θα πρέπει να περιλαμβάνει αραβικούς αριθμούς) για να γράψουμε την απάντηση σε μια ερώτηση ή μια λύση σε κάποια άσκηση. Στην περίπτωση αυτή η σύνταξή μας θα είναι όπως παρακάτω:
\begin{center}
\begin{tabular}{l}
\verb=\newanswer{solutions}=\\
\verb=\begin{exercises}[Ερωτήσεις]=\\
\verb=\item 1ο ερώτημα=\\
\verb=\begin{writesolutions}=\\
\verb=Εδώ γράφουμε την 1η απάντηση=\\
\verb=\end{writesolutions}=\\
\verb=\item 2ο ερώτημα=\\
\verb=\begin{writesolutions}=\\
\verb=Εδώ γράφουμε την 2η απάντηση=\\
\verb=\end{writesolutions}=\\
\verb=. . . . . . . . .=\\
\verb=\end{exercises}=
\end{tabular}
\end{center}
Όπως βλέπουμε στο περιβάλλον σημειώνουμε κανονικά κώδικα \LaTeX.  Ο κώδικας αυτός «γράφεται» σε ένα αρχείο
μαζί με τον αριθμό του προβλήματος. Επίσης σημειώνουμε τη λύση ακριβώς μετά το τέλος της εκφώνησης του
προβλήματος.
\item Όταν τελειώσουμε με τις λύσεις των προβλημάτων, πρέπει να «κλείσουμε» τα διάφορα αρχεία. Αυτό γίνεται
με την εντολή
 \begin{center}
      \Lcomm{close\textit{solutions}}
 \end{center}
Προσοχή! Όσες εντολές ορισμού νέου είδους απαντήσεων έχουμε, τόσες εντολές κλεισίματος πρέπει να υπάρχουν.
\item Τέλος, για να εμφανίσουμε τις απαντήσεις στο σημείο του βιβλίου που θέλουμε πχ. ένα παράρτημα του βιβλίου, γράφουμε την εντολή
\begin{center}
      \Lcomm{input\textit{solutions}}
      \end{center}
\end{enumerate}
Σε περίπτωση που επιθυμούμε οι απαντήσεις των ασκήσεων/ερωτήσεων να τοποθετηθούν στο τέλος κάθε κεφαλαίου
θα πρέπει να ορίσουμε για κάθε κεφάλαιο που έχει ασκήσεις ή ερωτήσεις ένα ξεχωριστό περιβάλλον
\verb|\newanswer| π.χ. οι ασκήσεις του κεφαλαίου 1 να ονομάζονται \verb|\newanswer{solutionsΙ}|, του
κεφαλαίου 2, \verb|\newanswer{solutionsΙΙ}| κοκ. Αφού γράψουμε τις ερωτήσεις και τις απαντήσεις τους,
θα πρέπει να κλείσουμε το περιβάλλον με την εντολή \Lcomm{close\textit{solutionsΙ}} κοκ., σε κάθε κεφάλαιο, και
στη συνέχεια  να τις εμφανίσουμε εντός του κεφαλαίου με την \Lcomm{input\textit{solutionsΙ}}.

Εφόσον θέλουμε οι ασκήσεις/ερωτήσεις να τοποθετούνται σε κάθε κεφάλαιο, αλλά το σύνολο των
απαντήσεων να συγκεντρώνεται και να παρουσιάζεται σε κάποιο άλλο μέρος του βιβλίου, πχ. σε κάποιο
παράρτημα, τότε, πριν την έναρξη της πρώτης ομάδας ασκήσεων στο βιβλίο μας ανοίγουμε ένα περιβάλλον
\verb|\newanswer| \verb|{όνομαπουεπιλέξαμε}| το οποίο και κλείνουμε μετά το τέλος της τελευταίας άσκησης
στο τελευταίο κεφάλαιο του βιβλίου μας όπου υπάρχουν ασκήσεις με την εντολή \Lcomm{close\textit{όνομαπουεπιλέξαμε}}. Τέλος, εμφανίζουμε στο τέλος του βιβλίου συγκεντρωμένες όλες τις λύσεις με την
\Lcomm{input\textit{όνομαπουεπιλέξαμε}}.

Μπορείτε να μελετήσετε την υλοποίηση του περιβάλλοντος των ασκήσεων στον κώδικα του κεφαλαίου \ref{chap:sample-chapter}.

\subsubsection{Βιβλιογραφία}

Στα βιβλία του Κάλλιπου η τοποθέτηση της βιβλιογραφίας συστήνεται να γίνεται στο τέλος κάθε
κεφαλαίου, γιατί συχνά τα κεφάλαια καταφορτώνονται και χρησιμοποιούνται ξεχωριστά. Για την
εισαγωγή της βιβλιογραφίας προτείνουμε τη χρήση του πακέτου \texttt{biblatex}.

Αρχικά συγκεντρώστε όλες τις βιβλιογραφικές πηγές σας σε ένα ενιαίο αρχείο \texttt{.bib} και τοποθετήστε
το στον φάκελο \textbf{chapters}. Για δική σας διευκόλυνση, σημειώστε στο αρχείο τις βιβλιογραφίες
ανά κεφάλαιο, βάζοντας σχετικά σχόλια. Στο τέλος το αρχείο \texttt{.bib} σας θα μοιάζει κάπως έτσι:
\begin{verbatim}
...

%----------------------------Κεφάλαιο 06-------------------
@Book{asyro,
 author = {Συρόπουλος, Απόστολος},
 title = {Ψηφιακή τυπογραφία με το \XeLaTeX},
 publisher = {Επίκεντρο},
 year = {2010},
 address = {Θεσσαλονίκη},
 isbn = {978-960-458-259-4}
 }
%----------------------------Κεφάλαιο 07-------------------
 @Book{pap,
 author = {Παπανικολάου, Γ. and Παλαιολόγου, Δ. and Κατσαρέλη, Ε. and Κατσίλα, Θ.
 and Τσαρουχά, Χ. and Τζέτη, Μ. and Λιλάκος, Κ. and Δούκισσας, Λ.},
 title = { Εργαστηριακές ασκήσεις γενετικής του ανθρώπου.},
 publisher = {Σύνδεσμος Ελληνικών Ακαδημαϊκών Βιβλιοθηκών},
 year = {2015},
 address = {Αθήνα},
 isbn = {978-960-603-310-0}
 }

...
\end{verbatim}

Στην συνέχεια φορτώστε το πακέτο biblatex θέτοντας τις ανάλογες παραμέτρους στο βασικό σας αρχείο
(master file). Στην περίπτωσή σας η \verb|{chapters/KOY_master}| θα παραπέμπει στο όνομα του
δικού σας αρχείου \texttt{.bib} στον φάκελο \textbf{chapters} αντί για το όνομα του αρχείου \verb|KOY_master| που αναφέρουμε στο παρακάτω παράδειγμα:
\begin{verbatim}
...
\usepackage[sorting=none, maxnames=5, style=numeric, bibstyle=numeric]{biblatex}
\DeclareLanguageMapping{english}{greek}
\defbibheading{biboption}{\section*{Βιβλιογραφία}}
\bibliography{chapters/KOY_master}
...
\end{verbatim}

Για να τοποθετήσουμε τη βιβλιογραφία στο τέλος κάθε κεφαλαίου θα πρέπει να χρησιμοποιήσουμε το περιβάλλον
\verb|refsection| στο επίπεδο του ξεχωριστού κεφαλαίου και στη συνέχεια να εμφανίσουμε τη βιβλιογραφία
με την \verb|\printbibliography[heading=biboption]|. Αυτό γίνεται όπως φαίνεται στο παρακάτω παράδειγμα:
\begin{verbatim}
\chapter{Συγγραφή με την κλάση kalliposstd }\label{chap:kallipos-std}
\begin{refsection}
..................................
\printbibliography[heading=biboption]
\end{refsection}
\end{verbatim}

Με τον τρόπο αυτό, στο τέλος του κεφαλαίου δημιουργείται ένα τμήμα βιβλιογραφίας που δεν είναι αριθμημένο.

Για να ετοιμάσουμε τη βιβλιογραφία θα πρέπει να εκτελέσουμε διαδοχικά τις παρακάτω εντολές
\begin{verbatim}
xelatex ονομα_master_file.tex
biber ονομα_master_file
xelatex ονομα_master_file.tex
\end{verbatim}
\section{Παραρτήματα και χρηστικές οδηγίες}
\subsection{Παραρτήματα}
Στο τέλος του βιβλίου μπορείτε να βάλετε όσα παραρτήματα θεωρείτε αναγκαία. Τα παραρτήματα μπορεί να
τοποθετηθούν σε ένα τελευταίο, ξεχωριστό μέρος του βιβλίου, ιδιαίτερα σε περιπτώσεις που είναι αρκετά.
Για να φτιάξετε ένα παράρτημα, δημιουργήστε ένα καινούριο αρχείο \texttt{.tex} και τοποθετήστε το
στον φάκελο \textbf{chapters} ονοματίζοντάς τo με τη μορφή Όνομα\_Κύριου\_Συγγραφέα\_appendix\_01.tex
(2,3, για το δεύτερο, τρίτο παράρτημα κοκ.).
Στο τέλος του βασικού σας αρχείου (master file) προσθέστε τα παραρτήματα μετά την εντολή \verb|\appendix| όπως φαίνεται
παρακάτω:
\begin{verbatim}
...
\part{Παραρτήματα}
\appendix
\begin{refsection}
\chapter{Πίνακες}
\section{Σύμβολα νουκλεοτιδίων}\label{noukl_symbols}
\begin{table}[ht] \centering \small
\caption[Πίνακας συμβόλων νουκλεοτιδίων]{Πίνακας συμβόλων που χρησιμοποιούνται στα αρχεία που περιέχουν νουκλεοτιδικές αλληλουχίες.}
\vspace{2mm}
\begin{tabular} {c l}
 \hline
	&	\\
\textbf{Σύμβολο νουκλεοτιδίου}	& \textbf{Επεξήγηση} \\
	&	\\
 \hline
A 	& 	A\\
C	& 	C\\
G 	& 	G \\
T	& 	T \\
U	& 	U\\
R 	& 	A ή G\\
Y & 		C, T ή U\\
K	&	G, T ή U\\
M	&	A ή C\\
S	&	C ή G\\
W	&	A, T ή U\\
B	&	Όχι A (δηλαδή C, G, T ή U)\\
D	&	Όχι C (δηλαδή A, G, T ή U)\\
H	&	Όχι G (δηλαδή A, C, T ή U)\\
V	&	Ούτε T ούτε U (δηλαδή A, C ή G)\\
N	&	A, C, G, T, U\\
X	&	Μεταμφιεσμένο\\
-	&	Χάσμα ασαφούς μήκους\\
\hline
\end{tabular}
\label{table:table_2_2}
\end{table}
\newpage
\section{Σύμβολα αμινοξέων}\label{aminoacid_symbols}
\begin{table}[ht] \centering \small
\caption[Πίνακας συμβόλων αμινοξέων]{Πίνακας συμβόλων που χρησιμοποιούνται στα αρχεία που περιέχουν αλληλουχίες αμινοξέων (πρωτεΐνες).}
\vspace{2mm}
\begin{tabular}  {c c l  p{6cm}}
 \hline
	&	&&\\
\textbf{Σύμβολο}	&\textbf{Συντομογραφία}	&\textbf{Ονομασία}&\textbf{Κωδικόνια}\\
	&	&&\\
\hline
	&	&&\\
A	&Ala	&Aλανίνη&GCT, GCC, GCA, GCG\\
B	&Asp ή Asn	&Aσπαρτικό οξύ (D) ή Aσπαραγίνη (N)& GAT, GAC - AAT, AAC\\
C	&Cys	&Κυστεΐνη&TGT, TGC\\
D	&Asp	&Aσπαρτικό οξύ&GAT, GAC\\
E	&Glu	&Γλουταμικό οξύ&GAA, GAG\\
F	&Phe	&Φαινυλαλανίνη&TTT, TTC\\
G	&Gly	&Γλυκίνη&GGT, GGC, GGA, GGG\\
H	&His	&Ιστιδίνη&CAT, CAC\\
I	&Ile	&Ισολευκίνη&ATT, ATC, ATA\\
J	&Leu ή Ile	&Λευκίνη (L) ή Iσολευκίνη (I)&TTA, TTG, CTT, CTC, CTA,CTG - ATT, ATC, ATA\\
K	&Lys	&Λυσίνη&AAA, AAG\\
L	&Leu	&Λευκίνη&TTA, TTG, CTT, CTC, CTA,CTG\\
M	&Met	&Mεθειονίνη&ATG\\
N	&Asn	&Aσπαραγίνη&AAT, AAC\\
O	&Pyl	&Πυρολυσίνη&\\
P	&Pro	&Προλίνη&CCT, CCC, CCA, CCG\\
Q	&Gln	&Γλουταμίνη&GGT, GGC, GGA, GGG\\
R	&Arg	&Aργινίνη&CGT, CGC, CGA, CGG, AGA, AGG\\
S	&Ser	&Σερίνη&TCT, TCC, TCA, TCG, AGT, AGC\\
T	&Thr	&Θρεονίνη&ACT, ACC, ACA, ACG\\
U	&Sec	&Σεληνοκυστεΐνη&\\
V	&Val	&Βαλίνη&GTT, GTC, GTA, GTG\\
W	&Trp	&Tρυπτοφάνη&TGG\\
Y	&Tyr	&Tυροσίνη&TAT, TAC\\
Z	&Glu ή Gln	&Γλουταμικό οξύ (E) ή Γλουταμίνη (Q)&GAA, GAG - CAA, CAG\\
X	&	&Οποιοδήποτε αμινοξύ&\\
*	&STOP	&Τερματισμός μετάφρασης&TAA, TGA, TAG\\
-	&	&Χάσμα ασαφούς μήκους&\\
	&	&&\\
\hline
\end{tabular}
\label{table:table_2_3}
\end{table} 
\end{refsection}

\chapter{Απαντήσεις ερωτήσεων - Λύσεις ασκήσεων}
\begin{refsection}
\section{Κεφάλαιο 7}\label{sol_CH-7}
\subsection{Λύσεις ασκήσεων}
\inputsolutionsI
\subsection{Εργασίες}
\inputsolutionsII
\end{refsection}

\end{document}
\end{verbatim}
\subsection{Χρηστικές οδηγίες}
\subsubsection{Πίνακας περιεχομένων}
\paragraph{Εμφάνιση τίτλων στον πίνακα περιεχομένων:}
\begin{enumerate}
\item Σε ορισμένες περιπτώσεις μπορεί να χρειαστεί να
μεταβάλετε τον τίτλο του κεφαλαίου ή της ενότητας στον πίνακα περιεχομένων. Αυτό
γίνεται απλώς τοποθετώντας τον τίτλο του πίνακα περιεχομένων σε αγκύλες πριν τον τίτλο του
κεφαλαίου ή της ενότητας. Για παράδειγμα:
\begin{verbatim}
\section[Αυτή η εκδοχή του τίτλου θα εμφανιστεί στον πίνακα περιεχομένων]
{Αυτή η εκδοχή θα εμφανίζεται σαν επικεφαλίδα στο τμήμα του κεφαλαίου}
\end{verbatim}
\item Τίτλοι επικεφαλίδων που είναι μακροσκελείς μπορεί να χρειαστεί να χωριστούν σε περισσότερες από μία
γραμμές και μάλιστα στη λέξη που θέλετε χρησιμοποιώντας δύο αντιπαράλληλες μετά την εντολή \verb=\protect= όπως παρακάτω:
\begin{verbatim}
\section{Πρώτη γραμμή στον τίτλο \protect \\ λέξεις στη δεύτερη γραμμή}
\end{verbatim}
%\item Τίτλοι στον πίνακα περιεχομένων που είναι μακροσκελείς μπορούν να αλλάξουν γραμμή ακολουθώντας το
%παρακάτω παράδειγμα:
%\begin{verbatim}
%\section[Εδώ βλέπουμε πώς μπορούμε να αλλάξουμε γραμμή
%\string\hfill\string\break\space στον πίνακα περιεχομένων]
%{Τίτλος της ενότητας του κεφαλαίου}
%\end{verbatim}
\end{enumerate}

\subsubsection{Αναδίπλωση τίτλων στο κείμενο}
\begin{enumerate}
\item Τίτλοι επικεφαλίδων που είναι μακροσκελείς μπορεί να χρειαστεί να χωριστούν σε περισσότερες από μία
γραμμές και μάλιστα στη λέξη που θέλετε χρησιμοποιώντας δύο αντιπαράλληλες ακολουθώντας το παρακάτω παράδειγμα κώδικα:
\begin{verbatim}
\subsubsection[Διαταραχές οξεοβασικής ισορροπίας 
και ηλεκτρολυτών στην χρόνια νεφρική νόσο]{%
\begin{tabular}[t]{l} Διαταραχές οξεοβασικής ισορροπίας \\ 
και ηλεκτρολυτών στη \\ 
χρόνια νεφρική νόσο \end{tabular}} 
\end{verbatim}
%\item Τίτλοι στον πίνακα περιεχομένων που είναι μακροσκελείς μπορούν να αλλάξουν γραμμή ακολουθώντας το
%παρακάτω παράδειγμα:
%\begin{verbatim}
%\section[Εδώ βλέπουμε πώς μπορούμε να αλλάξουμε γραμμή
%\string\hfill\string\break\space στον πίνακα περιεχομένων]
%{Τίτλος της ενότητας του κεφαλαίου}
%\end{verbatim}
\end{enumerate}
%\vspace{3mm}
\subsubsection{Πίνακες, εξισώσεις και κώδικας}\label{subsub:tables}
\paragraph{Πίνακες:}
Προσπαθήστε να κρατήσετε τους πίνακές σας λιτούς. Αποφύγετε τα εφέ και τους χρωματισμούς. Αν αποφασίσετε
ότι τα χρώματα είναι απολύτως απαραίτητα, υπολογίστε ότι πολλοί αναγνώστες θα εκτυπώνουν τις σελίδες
ασπρόμαυρα. Στους πίνακες του παρόντος Οδηγού έχουμε αποφύγει την τοποθέτηση πολλών γραμμών για να
κάνουμε τους πίνακες λιγότερο κουραστικούς. Αν έχετε μεγάλους και πολύπλοκους πίνακες σκεφτείτε την ιδέα να
τους αναπτύξετε σε ένα ξεχωριστό αρχείο .tex ώστε να μην επιβαρύνετε τον κώδικά σας και να μην δυσκολεύεστε
στην αποσφαλμάτωσή του. Στην συνέχεια μπορείτε να τους σώσετε ως αρχεία pdf τα οποία θα εντάξετε
στο βιβλίο σας. Πχ.
\begin{verbatim}
\includegraphics[width=1\textwidth]{images/table.pdf}
\end{verbatim}
Στην περίπτωση αυτή μην ξεχάσετε να αντιγράψετε το αρχείο pdf στον φάκελο \textbf{tables} (εκτός από τον φάκελο \textbf{images}) κατά την παράδοση του υλικού.
\paragraph{Εξισώσεις:}
Κατά τη χρήση του περιβάλλοντος texttt{equation} αποφύγετε να αφήνετε κενές γραμμές πριν και
μετά το περιβάλλον:
\begin{verbatim}
Δεν υπάρχει κενή γραμμή που να χωρίζει το περιβάλλον equation από το κείμενο
\begin{equation}
\alpha\beta\Gamma\Delta
\end{equation}
Ούτε από κάτω υπάρχει κενή γραμμή...
\end{verbatim}
\paragraph{Κώδικας:}
Χρησιμοποιήστε κατά προτίμηση το περιβάλλον \texttt{lstlisting}, αφού φορτώσετε το αντίστοιχο πακέτο (\verb|\usepackage{listings}|):

\subsubsection{Δημιουργία του εξώφυλλου και του οπισθόφυλλου}
Αφού δημιουργήσετε το εξώφυλλο του βιβλίου με τη βοήθεια του γραφιστικού επιμελητή θα χρειαστεί να το
μετατρέψετε σε αρχείο pdf. Συστήνεται το αρχείο αυτό να περιέχει 4 σελίδες ως εξής:
\begin{description}
\item [1η σελίδα] Το εξώφυλλο του βιβλίου
\item [2η σελίδα] Κενή δεύτερη σελίδα του βιβλίου
\item [3η σελίδα] Κενή προ-τελευταία σελίδα του βιβλίου
\item [4η σελίδα] Οπισθόφυλλο του βιβλίου
\end{description}
Τοποθετήστε το αρχείο στον φάκελο \textbf{images} με την ονομασία Όνομα-κύριου-συγγραφέα\_cover.pdf. Για την εισαγωγή του εξώφυλλου στο \XeLaTeX\ χρησιμοποιείστε την εντολή:
\begin{verbatim}
\includepdf[pages=1-2]{images/cover.pdf}
\end{verbatim}
Τοποθετήστε την εντολή στην αρχή του βιβλίου.
Για την εισαγωγή του οπισθόφυλλου χρησιμοποιήστε την εντολή:
\begin{verbatim}
\includepdf[pages=3-4]{images/cover.pdf}
\end{verbatim}
στο τέλος του βιβλίου.

\subsubsection{Περισσότερες πληροφορίες}
Για έναν αναλυτικό Οδηγό συγγραφής με \XeLaTeX\ μπορείτε να ανατρέξετε στο ελληνόγλωσσο σύγγραμμα του
Α. Συρόπουλου \textit{<<Ψηφιακή τυπογραφία με το \XeLaTeX>>}\cite{asyro}.

\printbibliography[heading=biboption]
\end{refsection}
