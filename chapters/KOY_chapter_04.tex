\chapter{Γλωσσικός Οδηγός προς Συγγραφείς - Συγγραφικές Ομάδες}\label{chap:lang-guide}
\chapterauthor{Κεντρική Ομάδα Υποστήριξης\\  Κάλλιπος, Ανοικτές Ακαδημαϊκές Εκδόσεις}

\section{Γενικές οδηγίες}
Κατά τη χρήση αντωνυμιών, βεβαιωθείτε ότι είναι σαφές σε ποιον όρο γίνεται η
αναφορά. Προτιμήστε τη χρήση αντωνυμιών στα διάφορα γένη (ο οποίος, η οποία, το
οποίο), αριθμούς (οι οποίοι, οι οποίες, τα οποία) και πτώσεις, αντί του αναφορικού που,
καθώς η έκφραση των νοημάτων και η συσχέτιση με τον όρο αναφοράς γίνεται με
μεγαλύτερη ακρίβεια.
\subsection{Ορολογία στην ελληνική γλώσσα ή/και σε άλλες γλώσσες}
Η χρήση των όρων πρέπει να γίνεται με βάση τα ακόλουθα:
\begin{itemize}
\item Κατά την πρώτη αναφορά ακρωνυμίου ή αρκτικόλεξου, γράψτε τον όρο αναλυμένο,
ακολουθούμενο μέσα σε παρένθεση από το ακρωνύμιο ή αρκτικόλεξό του:
Πχ. Ο Σύνδεσμος Ελληνικών Ακαδημαϊκών Βιβλιοθηκών (ΣΕΑΒ) συγκαλεί τα μέλη
του σε Γενική Συνέλευση δύο φορές τον χρόνο.
Είτε χρησιμοποιείτε τελείες στο ακρωνύμιο/αρκτικόλεξο είτε όχι (συστήνεται!),
τηρήστε με συνέπεια την επιλογή σας σε όλη την έκταση του κειμένου.
Σε προτάσεις που τελειώνουν με αρκτικόλεξα/ακρωνύμια για τα οποία χρησιμοποιείτε
τελεία, μην την επαναλαμβάνετε για να δηλώσετε ότι ολοκληρώνεται η περίοδος του λόγου:
Πχ. Ο συντονισμός των εκδηλώσεων γίνεται από τον Σ.Ε.Α.Β.
\item Κατά την πρώτη αναφορά ελληνικού όρου του οποίου μεταφραστικό ισοδύναμο υπάρχει
σε άλλη γλώσ\-σα, οι δύο όροι θα πρέπει να εμφανίζονται μαζί (με τον ξενόγλωσσο όρο
να ακολουθεί μέσα σε παρένθεση). Πχ. Η πολυτροπική επισημείωση (multimodal annotation) δίνει τη δυνατότητα
επισημείωσης σε υλικό που περιέχει εικόνα, βίντεο, ήχο, είτε συνδυασμούς των
προαναφερθέντων μέσων.

Σε περίπτωση που ξενικός όρος αντιστοιχεί σε περισσότερους του ενός ελληνικούς όρους
(ή το αντίστροφο), τηρήστε με συνέπεια την επιλογή που θα κάνετε. Πχ., ο όρος name
identities αποδίδεται ως ονοματικές οντότητες, ονόματα οντοτήτων κλπ. Η επιλογή σας θα
πρέπει να είναι μία και να μην εμφανίζονται πολλαπλές αποδόσεις που μπορεί να
δημιουργήσουν σύγχυση.
\item Σε περίπτωση που η ίδια έννοια εκφράζεται με όρους που ο ένας επικαλύπτει ή έχει
αντικαταστήσει τον άλλον, δηλώστε ποιος είναι ο όρος που χρησιμοποιείται στην
τρέχουσα επιστημονική ορολογία, κάνοντας, όμως, αναφορά και σ’ αυτούς που
χρησιμοποιούνταν στο παρελθόν. Για παράδειγμα: Σακχαρώδης Διαβήτης τύπου 1 /
Diabetes mellitus type 1 (γνωστός και ως διαβήτης τύπου 1 / type 1 diabetes, ο οποίος μέχρι
πρότινος αναφερόταν ως ινσουλινοεξαρτώμενος διαβήτης / insulin dependent diabetes ή
νεανικός διαβήτης / juvenile diabetes).
\item Τέλος, παραθέστε, μετά τα περιεχόμενα, στην αρχή του συγγράμματος, έναν Πίνακα με
όλα τα αρκτικόλεξα/ακρωνύμια που χρησιμοποιήσατε (στην ελληνική και στην αγγλική
γλώσσα), αναλυμένα (ολογράφως). Εναλλακτικά, συντάξτε έναν Πίνακα όπου θα
αναφέρονται αντιστοιχισμένοι οι ελληνικοί όροι με τους όρους στις άλλες γλώσσες και
ενσωματώστε τον στο Ευρετήριο (index) ή παραθέστε τον μετά, στο τέλος του
συγγράμματος.
\end{itemize}
\subsection{Κύρια Ονόματα}
Είναι προτιμότερο να διατηρείτε τα ανθρωπωνύμια και τοπωνύμια στη γλώσσα
προέλευσής τους (πχ. Steve Jobs, Stéphane Mallarmé), με εξαίρεση τις περιπτώσεις για τις
οποίες υπάρχει καθιερωμένη γραφή στην ελληνική γλώσσα (πχ. Λονδίνο, Σαίξπηρ κλπ.).
Σε περίπτωση που χρειαστεί (ή αποφασίσετε) να τα μεταγράψετε, ακολουθήστε τις οδηγίες
που δίνονται στο \emph{Διοργανικό Εγχειρίδιο Σύνταξης Κειμένων}
(\url{https://publications.europa.eu/code/el/el-4100500el.htm}).

Κατά την αναφορά σε χώρες και νομίσματα (κυρίως στις συντομογραφίες τους),
εφαρμόστε ό,τι προτείνει το \emph{Διοργανικό Εγχειρίδιο Σύνταξης Κειμένων}
(\url{http://publications.europa.eu/code/el/el-
370100.htm,http://publications.europa.eu/code/el/el-370303.htm#code})
ή υιοθετήστε κάποιο από τα ισχύοντα ISO (πχ. 639-2).

\subsection{Λίστες και απαρίθμηση}
Κατά τη χρήση λίστας, υιοθετήστε την ακόλουθη λογική:
\begin{itemize}
\item Σε απαρίθμηση ονομάτων ή φράσεων, ξεκινήστε με άνω και κάτω τελεία, γράψτε το
πρώτο γράμμα μικρό, χωρίστε με κόμμα και βάλτε τελεία μετά το τελευταίο στοιχείο:
\begin{itemize}
\item ένα,
\item δύο,
\item τρία.
\end{itemize}

{ή}

\begin{itemize}
\item πρώτο στοιχείο,
\item δεύτερο στοιχείο,
\item τρίτο στοιχείο.
\end{itemize}
\item Σε απαρίθμηση ολοκληρωμένων προτάσεων, ξεκινήστε κάθε πρόταση με κεφαλαίο
και στο τέλος της βάλτε τελεία.
\begin{itemize}
\item Εδώ αρχίζει η λίστα.
\item Εδώ συνεχίζεται η λίστα.
\item Εδώ ολοκληρώνεται η λίστα.
\end{itemize}
\end{itemize}
\textbf{Σημείωση:} Για Καλές Πρακτικές Συγγραφής, συστήνεται, προς αξιοποίηση, ο «Οδηγός
Γλωσσικής Επιμέλειας» της Δράσης ΚΑΛΛΙΠΟΣ (β’έκδοση \footnote{Γενικά, μεταβαίνοντας στη διεύθυνση \url{https://publications.europa.eu/code/el/el-4100000.htm} (Τελευταία ενημέρωση: 28.2.2019) δηλ. σε οδηγίες για την παρουσίαση των κειμένων στην ελληνική γλώσσα, αποκτάτε πρόσβαση στις αντίστοιχες ενότητες του Διοργανικού εγχειρίδιου, κάποιες από τις οποίες ενδέχεται -σε ορισμένα σημεία- να έχουν επικαιροποιηθεί σε σχέση με τα αναγραφόμενα στον ως άνω Οδηγό της Δράσης. Όσον αφορά δύο (2) υπερσυνδέσμους του Οδηγού, οι οποίοι στη β’ έκδοσή του δεν «ανοίγουν», παρατίθενται εδώ επικαιροποιημένοι, για ενδεχόμενη χρήση τους
[\url{http://ebooks.edu.gr/ebooks/v/html/8547/2009/Grammatiki\_E-ST-Dimotikou\_html-apli/},
\url{https://eeyem.eap.gr/wp-content/uploads/2016/09/APA\_ver2.pdf} (αφορά την 6η έκδοση APA), ενώ
σημειώνεται πως πρόσφατα η APA έχει εκδώσει και 7η έκδοση, ενδεικτικά
βλ. \url{https://apastyle.apa.org/style-grammar-guidelines/references/examples}].} ,
\url{http://repository.seab.gr/bitstream/1/27/5/20150728\_Glossikis\_Epimeleias\_v2.pdf}), όπου
παρέχονται αναλυτικές Οδηγίες/Υποδείξεις για ζητήματα γλώσσας/περιεχομένου που θα
αντιμετωπίσουν και θα κληθούν να επιλύσουν τα μέλη των Συγγραφικών Ομάδων, καθώς
και οι Συντελεστές επικουρικών εργασιών.
