\schapter{Πρόλογος} % Με την εντολή αυτή δεν έχουμε αρίθμηση αλλά μπαίνει και στον πίνακα περιεχομένων.
\chapterauthor{Κεντρική Ομάδα Υποστήριξης}[Κεντρική Ομάδα Υποστήριξης\\ Κάλλιπος, Ανοικτές Ακαδημαϊκές Εκδόσεις\\ \texttt{https://helpdesk.kallipos.gr}]


        % Αυτή η εντολή χρησιμοποιείται αν έχουμε διαφορετικό συγγραφέα
        % από τον κύριο συγγραφέα του βιβλίου ή δημιουργούμε μια ανθολογία
        % Το δεύτερο optional όρισμα υπάρχει αν θέλουμε να βάλουμε επιπλέον
        % πληροφορίες στον τίτλο του κεφαλαίου αλλά όχι στα περιεχόμενα.

Πρόλογος  θεωρείται το προεισαγωγικό τμήμα του βιβλίου, που τοποθετείται πριν από την εισαγωγή (εφόσον υπάρχει).

Ο πρόλογος είναι ένα πολύ σύντομο κείμενο, συχνά γραμμένο από πρόσωπο διαφορετικό από τον συγγραφέα (από τον εκδότη, από γνωστό ειδικό, από πρόσωπο συνδεόμενο με τον συγγραφέα κ.λπ.). Ο πρόλογος ενίοτε γράφεται και από τον ίδιο τον συγγραφέα, για να περιλάβει σ’ αυτόν γενικότερες πληροφορίες, τεχνικά θέματα του βιβλίου, ευχαριστίες κ.λπ. Εφόσον υπάρχει εισαγωγή, ο πρόλογος προηγείται της εισαγωγής και διαφοροποιείται απ’ αυτήν πλήρως. Εάν δεν υπάρχει εισαγωγή, ο πρόλογος περιέχει στοιχεία που κανονικά θα δίνονταν σε μια εισαγωγή και είναι εκτενέστερος.
