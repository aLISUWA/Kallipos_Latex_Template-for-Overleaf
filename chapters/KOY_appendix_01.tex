\begin{refsection}
\chapter{Πίνακες}
\section{Σύμβολα νουκλεοτιδίων}\label{noukl_symbols}
\begin{table}[ht] \centering \small
\caption[Πίνακας συμβόλων νουκλεοτιδίων]{Πίνακας συμβόλων που χρησιμοποιούνται στα αρχεία που περιέχουν νουκλεοτιδικές αλληλουχίες.}
\vspace{2mm}
\begin{tabular} {c l}
 \hline
	&	\\
\textbf{Σύμβολο νουκλεοτιδίου}	& \textbf{Επεξήγηση} \\
	&	\\
 \hline
A 	& 	A\\
C	& 	C\\
G 	& 	G \\
T	& 	T \\
U	& 	U\\
R 	& 	A ή G\\
Y & 		C, T ή U\\
K	&	G, T ή U\\
M	&	A ή C\\
S	&	C ή G\\
W	&	A, T ή U\\
B	&	Όχι A (δηλαδή C, G, T ή U)\\
D	&	Όχι C (δηλαδή A, G, T ή U)\\
H	&	Όχι G (δηλαδή A, C, T ή U)\\
V	&	Ούτε T ούτε U (δηλαδή A, C ή G)\\
N	&	A, C, G, T, U\\
X	&	Μεταμφιεσμένο\\
-	&	Χάσμα ασαφούς μήκους\\
\hline
\end{tabular}
\label{table:table_2_2}
\end{table}
\newpage
\section{Σύμβολα αμινοξέων}\label{aminoacid_symbols}
\begin{table}[ht] \centering \small
\caption[Πίνακας συμβόλων αμινοξέων]{Πίνακας συμβόλων που χρησιμοποιούνται στα αρχεία που περιέχουν αλληλουχίες αμινοξέων (πρωτεΐνες).}
\vspace{2mm}
\begin{tabular}  {c c l  p{6cm}}
 \hline
	&	&&\\
\textbf{Σύμβολο}	&\textbf{Συντομογραφία}	&\textbf{Ονομασία}&\textbf{Κωδικόνια}\\
	&	&&\\
\hline
	&	&&\\
A	&Ala	&Aλανίνη&GCT, GCC, GCA, GCG\\
B	&Asp ή Asn	&Aσπαρτικό οξύ (D) ή Aσπαραγίνη (N)& GAT, GAC - AAT, AAC\\
C	&Cys	&Κυστεΐνη&TGT, TGC\\
D	&Asp	&Aσπαρτικό οξύ&GAT, GAC\\
E	&Glu	&Γλουταμικό οξύ&GAA, GAG\\
F	&Phe	&Φαινυλαλανίνη&TTT, TTC\\
G	&Gly	&Γλυκίνη&GGT, GGC, GGA, GGG\\
H	&His	&Ιστιδίνη&CAT, CAC\\
I	&Ile	&Ισολευκίνη&ATT, ATC, ATA\\
J	&Leu ή Ile	&Λευκίνη (L) ή Iσολευκίνη (I)&TTA, TTG, CTT, CTC, CTA,CTG - ATT, ATC, ATA\\
K	&Lys	&Λυσίνη&AAA, AAG\\
L	&Leu	&Λευκίνη&TTA, TTG, CTT, CTC, CTA,CTG\\
M	&Met	&Mεθειονίνη&ATG\\
N	&Asn	&Aσπαραγίνη&AAT, AAC\\
O	&Pyl	&Πυρολυσίνη&\\
P	&Pro	&Προλίνη&CCT, CCC, CCA, CCG\\
Q	&Gln	&Γλουταμίνη&GGT, GGC, GGA, GGG\\
R	&Arg	&Aργινίνη&CGT, CGC, CGA, CGG, AGA, AGG\\
S	&Ser	&Σερίνη&TCT, TCC, TCA, TCG, AGT, AGC\\
T	&Thr	&Θρεονίνη&ACT, ACC, ACA, ACG\\
U	&Sec	&Σεληνοκυστεΐνη&\\
V	&Val	&Βαλίνη&GTT, GTC, GTA, GTG\\
W	&Trp	&Tρυπτοφάνη&TGG\\
Y	&Tyr	&Tυροσίνη&TAT, TAC\\
Z	&Glu ή Gln	&Γλουταμικό οξύ (E) ή Γλουταμίνη (Q)&GAA, GAG - CAA, CAG\\
X	&	&Οποιοδήποτε αμινοξύ&\\
*	&STOP	&Τερματισμός μετάφρασης&TAA, TGA, TAG\\
-	&	&Χάσμα ασαφούς μήκους&\\
	&	&&\\
\hline
\end{tabular}
\label{table:table_2_3}
\end{table} 
\end{refsection}
