\chapter{Καλές πρακτικές συγγραφής - Παράδοση υλικού}\label{chap:best-practice}
\chapterauthor{Κεντρική Ομάδα Υποστήριξης\\  Κάλλιπος, Ανοικτές Ακαδημαϊκές Εκδόσεις}

\section{Καλές πρακτικές συγγραφής}
\subsection{Γενικές οδηγίες}
Εφόσον κάθε Συγγραφέας έχει το δικό του προσωπικό ύφος γραφής, οι παρακάτω οδηγίες
περιορίζονται απλώς στην επισήμανση καλών πρακτικών που θα ήταν χρήσιμο να
ληφθούν υπόψη κατά τη διάρκεια της συγγραφής, δεδομένων και των ιδιαιτεροτήτων ενός
ηλεκτρονικού βιβλίου:
\begin{itemize}
\item Η ηλεκτρονική μορφή του βιβλίου δίνει τη δυνατότητα (άμεσης) μεταφοράς στο
σημείο στο οποίο κάποιος θέλει να παραπέμψει μέσα στο κείμενό του, οπότε η
παρουσίαση του υλικού μπορεί να γίνει χωρίς την (περιττή) επανάληψη
πληροφοριών.
\item Κατά τη μεταφορά υλικού από άλλη πηγή, όταν γίνεται αντιγραφή και
επικόλληση, θα πρέπει να δίνεται προσοχή για να μην
μεταφέρονται στο κείμενο εντολές στοιχειοθέτησης που μπορεί να χαλάσουν την
ήδη υφιστάμενη. Επίσης, συστήνεται να ελέγχεται η ορθογραφία του, καθώς είναι
συχνό φαινόμενο να μεταφέρονται λάθη, όπως και χαρακτήρες που δημιουργούν
πρόβλημα.
\end{itemize}
Γενικότερα, να αποφεύγεται η γραφή με αποκλειστική χρήση κεφαλαίων (κυρίως στις
επικεφαλίδες και στους τίτλους/υπότιτλους), καθώς κάτι τέτοιο δυσχεραίνει την
ανάγνωση, ενώ ορισμένες φορές δημιουργεί και αμφισημία.
\subsection{Καλές πρακτικές συγγραφής \XeLaTeX}
Κατά τη συγγραφή σε \XeLaTeX\ είναι καλό να μην γράφουμε μεγάλες γραμμές κώδικα γιατί στη
συνέχεια δυσχεραίνεται η αποσφαλμάτωσή του. Αρκετοί επεξεργαστές διαθέτουν λειτουργικότητα
που βοηθά τη συγγραφή. Συνήθως 80-100 χαρακτήρες ανά γραμμή είναι αρκετοί.

Κατά τη συγγραφή σημαντικό είναι να εισάγουμε κατατοπιστικά σχόλια ώστε να μπορούμε
με ευκολότερο τρόπο να ανατρέξουμε στις περιοχές του κώδικα που επιθυμούμε.

Προσπαθούμε πάντα να διατηρήσουμε των αριθμό των πακέτων που φορτώνονται στα απολύτως
απαραίτητα για τη σωστή λειτουργικότητα του κειμένου μας.

\section{Υλικό προς παράδοση}\label{package}

Στο Πρωτογενές Υλικό της συγγραφικής δραστηριότητας περιλαμβάνονται: αρχεία
κειμένου, εικόνες, σχήματα, βίντεο, διαδραστικά αντικείμενα, αριθμητικά δεδομένα
παραγωγής σχημάτων (datasets) κλπ.

Αναλυτικά, το Πρωτογενές Υλικό θα παραδίδεται σε ξεχωριστούς φακέλους ως εξής:
\begin{itemize}
\item \textbf{Φάκελος “01\_text”}: Εδώ εντάσσεται το συνολικό αρχείο ή/και τα αρχεία του
κειμένου (για τις περιπτώσεις ξεχωριστών αρχείων ανά Κεφάλαιο). Το όνομα του
κάθε αρχείου θα δίνεται ως εξής: <Ε\-πώ\-νυ\-μο Κύριου Συγγραφέα>\_master (για τις
περιπτώσεις ενός αρχείου, πχ. Papadopoulos\_master.docx) ή <Επώ\-νυ\-μο Κύριου
Συγγραφέα>\_chapter\_n (Papadopoulos\_chapter\_01.docx), όπου n θα είναι ο
αντίστοιχος αριθμός του κάθε Κεφαλαίου. Ο συνηθισμένος αποδεκτός μορφότυπος
κειμένου είναι το docx. Δεν αφορά τους συγγραφείς \LaTeX.
\item \textbf{Φάκελος “02\_tex”}: Ο εν λόγω φάκελος αφορά αποκλειστικά Συγγραφείς που
συγγράφουν σε \LaTeX, \XeLaTeX. Τα αρχεία .tex θα πρέπει να είναι οργανωμένα ως
εξής: ένα κύριο αρχείο .tex με όνομα <Ε\-πώ\-νυ\-μο Κύριου Συγγραφέα>\_master.tex
το οποίο θα περιλαμβάνει αναφορές σε ξεχωριστά αρχεία .tex, ένα για κάθε
Κεφάλαιο. Στον υποφάκελο "chapters" θα πρέπει να συμπεριλαμβάνονται τα
ξεχωριστά αρχεία .tex για κάθε Κεφάλαιο με ονομασία <Επώνυμο Κύριου
Συγγραφέα>\_chapter\_n (Papadopoulos\_chapter\_01.tex). Στον υποφάκελο "images" τα επεξεργασμένα τελικά αρχεία εικόνων που ενσωματώνονται στα αρχεία .tex. Σε κάθε περίπτωση, το
σύνολο των αρχείων θα πρέπει να επιτρέπει τη μετατροπή των αρχείων .tex στο
πλήρες PDF του συγγράμματος. Στον φάκελο .tex συμπεριλαμβάνεται
υποφάκελος PDF, όπου τοποθετούνται, τόσο το PDF του συγγράμματος, όσο και
αρχεία PDF για κάθε Κεφάλαιο ή Παράρτημα, ξεχωριστά. Η ονοματοδοσία των
αρχείων PDF θα πρέπει να ακολουθεί τους ίδιους κανόνες με την ονοματολογία των
αρχείων .tex. Οι συγγραφείς μπορούν απλώς να αντιγράψουν στον φάκελο “02\_tex” το περιεχόμενο του φακέλου που συγγράφουν εφόσον ακολουθούν το πρότυπο του Έργου.
\item \textbf{Φάκελος “03\_tables”}: Εδώ εντάσσονται οι Πίνακες σε μορφή Εικόνας εφόσον
αποτελούν μαθησιακό αντικείμενο, συμπεριλαμβανομένων και των μεταδεδομένων
(βλ. παράγραφο \ref{par:metadata}). Με εξαίρεση τυχόν μεταδεδομένα οι συγγραφείς \LaTeX, \XeLaTeX\ δεν χρειάζεται να συμπεριλάβουν Πίνακες σε αυτόν τον φάκελο, εκτός εάν έχουν ενσωματώσει στο βιβλίο τους Πίνακες με τη μορφή pdf (βλέπε σχετικά σελ. \pageref{subsub:tables}).
\item \textbf{Φάκελος “04\_images”}: Εδώ εντάσσονται τα Πρωτογενή (raw) αρχεία
Εικόνων/Σχημάτων κλπ. (για παράδειγμα, αρχεία Visio, Adobe Photoshop, Adobe
Illustrator ή οποιοδήποτε αρχείο στον μορφότυπο του αντίστοιχου λογισμικού στο
οποίο έγινε η επεξεργασία της Εικόνας) και τα αρχεία των Εικόνων/Σχημάτων όπως
αυτά εισάγονται στα προγράμματα επεξεργασίας κειμένου (MS Word ή \LaTeX)
(βλ. παράγραφο \ref{sec:images}). Τα Πρωτογενή αρχεία θα αποθηκεύονται στον υποφάκελο
raw\_images, ενώ τα παραγόμενα αρχεία που χρησιμοποιούνται για την εισαγωγή
τους στο κείμενο θα τοποθετούνται στον υποφάκελο final\_images. Τα περιεχόμενα του final\_images
ουσιαστικά είναι τα ίδια με τον υποφάκελο "images" του φάκελου "02\_tex" για τους συγγραφείς
\LaTeX, \XeLaTeX~. Σε κάθε περίπτωση η ονοματοδοσία όλων των αρχείων θα πρέπει να ακολουθεί τις
οδηγίες της παραγράφου \ref{package}.
\item \textbf{Φάκελος “05\_math”}: Δεν αφορά τους συγγραφείς \LaTeX, \XeLaTeX\,
\item \textbf{Φάκελος “06\_audio”}: Εδώ εντάσσονται τα αρχεία ήχου,
\item \textbf{Φάκελος “07\_video”}: Εδώ εντάσσονται τα αρχεία video,
\item \textbf{Φάκελος “08\_interactive”}: Εδώ εντάσσονται τα διαδραστικά αντικείμενα.
\end{itemize}

Όπου είναι απαραίτητο θα πρέπει να συμπεριλαμβάνεται και το αρχείο των μεταδεδομένων
ανά μαθησιακό αντικείμενο.

Σε κάθε περίπτωση, όταν για την προβολή ή και αναπαραγωγή του κάθε στοιχείου
(μαθησιακού αντικειμένου) χρειάζονται περισσότερα του ενός αρχεία (πχ. βιβλιοθήκες
javascript, αρχεία CSS), τότε όλα τα αρχεία θα δίνονται σε ένα συμπιεσμένο αρχείο (zip).
Η ΚΟΥ του Έργου μπορεί να αναλάβει τη μετατροπή σε ηλεκτρονική μορφή PDF, εφόσον
το Πρωτογενές Υλικό έχει μορφοποιηθεί όπως περιγράφεται στον παρόντα Οδηγό.
Όσον αφορά ειδικές περιπτώσεις, επιβάλλεται το να προηγείται συνεννόηση με την ΚΟΥ
για τον τρόπο και τη μορφή παράδοσης του τελικού υλικού.

Το σύνολο του υλικού θα τοποθετείται σε έναν φάκελο, το όνομα του
οποίου θα σχηματίζεται από το επώνυμο του κύριου Συγγραφέα και τις πρώτες λέξεις του
τίτλου, έως και τέσσερις λέξεις:\\
EPONYMO\_KYRIOU\_SYGRAFEA\_TITLOS\_SYGRAMATOS.zip.

Ο εν λόγω φάκελος θα συμπιέζεται σε ένα αρχείο zip (με το ίδιο όνομα με τον φάκελο), το οποίο και θα υποβληθεί ηλεκτρονικά. Για τη διευκόλυνσή σας μπορείτε να καταφορτώσετε συμπιεσμένο αρχείο με τη δομή των φακέλων που περιγράφεται παραπάνω από τον σύνδεσμο \href{https://www.kallipos.gr/images/kalliposplus/A4/EPONYMO_KYRIOU_SYGRAFEA_TITLOS_SYGRAMATOS.zip} {<<φάκελοι πρωτογενούς υλικού>>}.
